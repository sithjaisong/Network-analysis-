\subsection*{Network analysis}

Network models were used to illustrate the co-occurrence patterns of pest injuries within same locations, where injuries represent nodes and the presence of a co-occurrence relationship based on correlation is represented by an edge using the igraph package, where each network was the union of positive or negative correlation coefficients (less than −0.25 or greater than 0.25) that were consistent within each location. 

We were also interested in generating descriptive statistics about the network that may be important for understanding co-occurrence relationships. We produced network statistics that describe the position and connectedness of injuries within each co-occurrence network. Global network properties including the density, heterogeneity, centralization were computed by using \texttt{fundamentalNetworkConcepts} function from  WGCNA package and for the basic properties such as number of nodes, edges can be computed by using functions from igraph package. Addtionally, we also calculated the small-worldness index of the network by using \texttt{smallworldness} function qgraph package. For the node-wise properties including node degree, which is the number of co-occurrence relationships that an injury is involved in a network, we used the \texttt{degree} function from igraph package. We also calculated betweenness scores for each node (injury) using the \texttt{betweenness} function from igraph package, which is defined by the number of paths through a focal microbial node. Additionally, we calculated clustering coefficients, and eigenvector using the \texttt{transitivity} function for comparison to other networks.

