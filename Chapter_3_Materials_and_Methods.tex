\subsection{Materials and Methods}

\subsubsection{Study sites, data collection}

We survey the farmers' fields in five different production environments, which are Tamil Nadu  and Kerala, India; West Java, Indonesia; Laguna, Philippines; Central Plain, Thailand; and Mekong river delta, Vietnam for consecutive (2009 - 2013). These survey sites are in the major irrigated lowland rice growing areas, where rice is intensively cropped at least two seasons per year. The survey was conducted in both the rainy and dry seasons of 2010 to 2012 in India, Indonesia, Thailand, and Vietnam, except Philippines, which the survey was conducted in 2011. A total of 412 individual fields were surveyed in the five sites. The number of survey were summerize in Table \ref{table:Survey_data}. Each survey was regarded to be the representatives with a characteristic set of attributes, which were related to cropping practices, and environments.


The survey procedure and data were based on a standardized protocol described in ``A survey portfolio to characterize yield-reducing factors in rice'' developed by \citet{Savary_2009_Survey}. Rice injuries were collected in this study including the injuries caused by animal pests, pathogens, and weeds, which are harmful to rice plants, and importantly considered to reduce yield productivity. They were evaluated at booting and ripening stage according to survey procedure. Injuries on leaves such as whorl maggot injury (WM), leaffolder injury (LF), bacterial leaf blight (BLB), bacterial leaf streak (BLS), leaf blast (LB), brown spot (BS) were determined as a proportion of injured leaves. Injuries on tillers or hills such as stem rot (SR), sheath rot (SHR), sheath blight (SHB), whitehead (WH), deadheart (DH), Silver shoot (SS) were determined as a proportion of injured tillers or panicles. Weed infestation was determined as the percent of area are covered by any weed species above the rice canopy (WA) and the weed below the rice canopy (WB). Insect pests including brown planthopper (BPH), white backed planthopper (WPH), Green leafhopper (GLH), Rice bug or Stink bug (STB) were determined as number of insect found on the rice hill. 

Before analysis, data were compacted over time during crop growth. Two types of data were computed, depending on the nature of injury \citet{Savary_2009_Survey}. One is an area under injury progress curve (AUIPC) used for injury variables, which present on the leaves, and for weed infestation. Another is the maximum level at any of the two observations used for injury variables that can be observed on tillers, panicles, and hills, and insect pest count. The area under injury progress curve (AUIPC) \citep{Campbell_1990_Introduction} were calculated by the mid-point method using the following equation: The area under injury progress curve (AUIPC) was calculated by the midpoint method using the following equation::

\begin{equation}
AUIPC = \sum{\frac{1}{2(X_{i} + X_{i-1})(T_{i} - T_{i-1})}}
\end{equation}

where $X_i$ is percentage (\%) of leaves, tillers or panicles injured due to rice pests (e.g., leaf blast, leaf folder), or number of insects (e.g., plant hoppers, leaf hoppers) per quadrat, or percentage (\%) of weed infestation (ground coverage) at the $i$th observation, $T_i$ is time in rice development stage units (dsu) on a 0 to 100 scale (10: seedling, 20: tillering, 30: stem elongation, 40: booting, 50: heading, 60: flowering, 70: milk, 80: dough, 90: ripening, 100: fully mature) at the $i$th observation and $n$ is total number of observations.

\subsubsection{Network construction}

The co-occurrence network was inferred based on the Spearman correlation matrix constructed with \texttt{R} function \texttt{cor.test} with parameter method `Spearman' (package stats) was used for calculate Spearman's correlation coefficient ($\rho$), which is defined as the Pearson correlation coefficient between the ranked variables \cite{R_2015}. The nodes in this network represent injuries and the edges that connect these nodes represent correlations between injuries. Based on correlation coefficients and $P$-values for correlation, we constructed co-occurrence networks.The cutoff of $P$-values was 0.05. Network properties were calculated with the igraph package \citep{Csardi_2010_igraph}. All farmers' fields were divided into groups by country and season. The impact of each sample group on the Spearman correlation value of each edge in the network was assessed by Spearman correlation value of these fields. The network of each of group was detected the community structure by Walktrap algorithm by using \texttt{cluster_walktrap} function of \textbf{igraph} package.


\subsubsection{Topological feature analysis}

We calculated topological features at the network level and node level in the network with the \textit{igraph} package. The node level feature set included betweenness centrality (the number of shortest paths going through a node), closeness centrality (the number of steps required to access all other nodes from a given node), transitivity (the probability that the adjacent nodes of a node are connected, also called the clustering coefficient) and degree (the number of adjacent edges). The betweenness centrality feature was used to measure the centrality of each node in the network. Nodes were further identified the key factors in the network. The way to identify key factors is to compare relative values of centrality such as eigenvector centrality and betweenness \cite{Valente_2008_How}. The residual of linear relationship between eigenvector centrality and betweeness and regress betweeness on eigenvector centrality can used as an indicators of key factors. A node with higher levels of betweenness and lower eigenvector centrality can be inferred as a central of the functioning of the network. Nodes with lower levels of betweenness and higher eigenvector centrality mare recognized as a key to the functioning of the network. 

Erod\"{o}-R\'{e}nyi random networks were generated by the \textit{igraph} package for comparison to empirical networks, where for every empirical network with $n$ nodes and $m$ edges, a random network with $n$ nodes and $m$ edges was generated, with each edge having an equal probability of being assigned to any node. Network comparisons were made using connectance (degree),  average path length (average shortest path length between nodes) and clustering coefficient \citep{Kolaczyk_2014_Statistical}. These properties were calculated for real and random networks, and networks were visualized using \textit{igraph} package. 
 
\subsubsection{Statistical analysis}
The Spearman’s rank correlation test was used to examine the correlation between injuries and each topological feature.  To test for differences in topological features between country, we used the Kruska-Wallis H test in \textit{R}, and then performed multiple comparison test after Kruskal-Wallis using \textit{pgirmess} package \citep{Giraudoux_2012_pgirmess} and create homogenous group by using \textit{multcompView} package \citep{Spencer_2015_multcompView}.





