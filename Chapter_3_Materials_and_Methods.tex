\subsection{Materials and Methods}

\subsubsection{Study sites, data collection}

We surveyed farmers' fields in 5 different locations;Tamil Nadu, India (TMN); West Java; Indonesia (WJV), Laguna, Philippines (LAG); Suphanburi, Thailand (SPB) and Mekong river delta, Vietnam (MKD) for consecutive years (2009 - 2013). These survey sites are in the major irrigated lowland rice growing areas, where rice is intensively cropped at least two seasons per year. The 412 individual fields were surveyed in both dry and wet seasons of 2010 to 2012 in India, Indonesia, Thailand, and Vietnam, except Philippines, which the surveys were only conducted in 2011. The number of survey were summerize in Table 
\ref{table:Survey_data}.

The survey procedure and data were based on a standardized protocol described in ``A survey portfolio to characterize yield-reducing factors in rice'' developed by \citet{Savary_2009_Survey}. Rice injuries were collected in this study including the injuries caused by animal pests, and pathogens, which are harmful to rice plants, and importantly considered to reduce yield productivity. They were evaluated at booting and ripening stage according to survey procedure. Injuries on leaves such as whorl maggot injury (WM), leaffolder injury (LF), bacterial leaf blight (BLB), bacterial leaf streak (BLS), leaf blast (LB), brown spot (BS) were determined as a proportion of injured leaves. Injuries on tillers or hills such as stem rot (SR), sheath rot (SHR), sheath blight (SHB), whitehead (WH), deadheart (DH), gall midge injuries (GM) were determined as a proportion of injured tillers or panicles. Insect pests including brown planthopper (BPH), white backed planthopper (WPH), green leafhopper (GLH), rice bug (RB) or stink bug were determined as number of insect found on the rice hill. 

Before analysis, data were compacted over time during crop growth. Two types of data were computed, depending on the nature of injury \citet{Savary_2009_Survey}. One is an area under injury progress curve (AUIPC) used for injury variables, which present on the leaves, and for weed infestation. Another is the maximum level at any of the two observations used for injury variables that can be observed on tillers, panicles, and hills, and insect pest count. The area under injury progress curve (AUIPC) \citep{Campbell_1990_Introduction} were calculated by the mid-point method using the following equation: 

\begin{equation}
AUIPC = \sum{\frac{1}{2(X_{i} + X_{i-1})(T_{i} - T_{i-1})}}
\end{equation}

where $X_i$ is percentage (\%) of leaves, tillers or panicles injured due to rice pests (e.g., leaf blast, leaf folder), or number of insects (e.g., plant hoppers, leaf hoppers) per quadrat, or percentage (\%) of weed infestation (ground coverage) at the $i$th observation, $T_i$ is time in rice development stage units (dsu) on a 0 to 100 scale (10: seedling, 20: tillering, 30: stem elongation, 40: booting, 50: heading, 60: flowering, 70: milk, 80: dough, 90: ripening, 100: fully mature) at the $i$th observation and $n$ is total number of observations.

\subsubsection{Network construction}

The co-occurrence network was inferred based on the Spearman correlation matrix constructed with \texttt{R} function \texttt{cor.test} with parameter method `Spearman' (package stats) was used for calculate Spearman's correlation coefficient ($\rho$), which is defined as the Pearson correlation coefficient between the ranked variables \cite{R_2015}. The nodes in this network represent injuries and the edges that connect these nodes represent correlations between injuries. Based on correlation coefficients and $P$-values for correlation, we constructed co-occurrence networks.The cutoff of $P$-values was 0.05. Network properties were calculated with the \textbf{igraph} package \citep{Csardi_2010_igraph}. All farmers' fields were divided into groups by country and season. The impact of each sample group on the Spearman correlation value of each edge in the network was assessed by Spearman correlation value of these fields. The network of each of group was detected the community structures by maximizing the modularity measure over all possible partitions by using \texttt{cluster\_optimal} function of \textbf{igraph} package \cite{Brandes_2008_Modularity}.


\subsubsection{Topological feature analysis}

We calculated the topological features for each network with the \textit{igraph} package. We measured two levels of network topologies. The node features that we focused on are node degree, betweenness, and clustering coefficient. Node degree is measured by the number of the edges (connections) of a node has. Betweenness of a node is defined by the number of of shortest paths going through a node, and the local clustering coefficients of a node is the ratio of existing edges connecting a node's neighbors to each other to the maximum possible number of such edges.

Global features including network clustering coefficient, average path length, and diameter were measured for each network. The network clustering coefficient measures the degree to which nodes of the network tend to cluster together and is a measure of the connectedness of the network and is indicative of the degree of relationships in the network. Average path length is the average number of steps along the shortest paths for all possible pairs of network nodes, and diameter is the greatest distance between any pair of nodes. In our analysis, both diameter and average path length are considered measures of the size of the network. Larger networks are less connected, meaning that the likelihood of a strong connection between any two randomly selected species is low.
The network clustering coefficient are considered measures of the complexity of the network. The networks are more complex, the network has higher clustering coefficient, and shorter average path length.

Node were further classified by ranking all nodes according to three node features, partitioning this ranked list into three equally value of each node property. Nodes with high rank value in top third proportion of node degree, and high rank value in top third proportion of betweenness are recognized as indicator in co-occurrence network of rice injuries. 
\newpage