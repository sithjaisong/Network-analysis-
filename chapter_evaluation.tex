\section{Evaluation of correlation methods for co-occurrence network construction of rice crop health survey data}

Network analysis is a promising tool frequently used to describe the pairwise relationships of a large number of variables. Network analysis of pest and disease incidence co-occurrence patterns offers new insight into pest management. Since different correlation measures lead to structurally different co-occurrence network and provide different information, selecting the optimal correlation method is critical. Pests are the blocks in the big wall of yield losses that plant protection specialists must scale due to the need to increase crop productivity. Surveys of farmers’ fields are useful sources of data to help us determine the importance of pests and understand complex relationships within agroecosystems. Our objective is to find an appropriate method to ..

%general background information
% using the research ref to support both the background facts and claim for significance
% general problem or the current research focus
% 
\subsection{Materials and methods}
\textbf{Survey data}
Survey data collected in 420 farmers’ fields in irrigated lowland rice growing areas across South and Southeast Asia (Tamil Nadu; India, Odisha; India, West Java; Indonesia, Central plain; Thailand, and Red river delta; Vietnam) were based on the “Survey Portfolio” . In this study, we evaluated correlation measures including, Pearson, Spearman, Kendal and Biweight mid correlation to discover the true functionally related variables in crop health survey data.

\subsection{Result}
Evaluation of four correlation measures, including Pearson, Spearman, Kendall correlation, and Biweight mid-correlation shows two groups clustered according to hierarchical clustering using Euclidean distance. The Spearman and Kendall correlation were grouped in rank-base methods, and another group is non-rank-based correlations including Pearson correlation and Biweight mid-correlation. When choosing a measure, Spearman correlation was selected because its robustness to noise and to outliers.
Briefly, network analysis revealed that brown spot is the highest connection. Bacterial leaf streak, brown spot, white-backed planthopper, and brown planthopper have high node of connectivity

For example, a clustering method focuses on the analysis of (dis)similarities, whereas principal component analysis (PCA) attempts to explain as much variation as possible in as few components as possible. Changing data properties using data pretreatment may therefore enhance the results of a clustering method, while obscuring the results of a PCA analysis.

\textbf{Heteroscedasticity}
To determine the presence or absence of heteroscedasticity in the data set, the standard deviations of the metabo- lites of the analytical and the biological repeats were analyzed (Figure 4). Analysis of the analytical and the uninduced biological standard deviations showed that heteroscedasticity was present both in the analytical error and in the biological uninduced variation (Figure 4A and 4B). In contrast, the relative biological standard deviation (Figure 4C), and also the relative analytical standard devi- ation (unpublished results), showed the opposite effect. Thus, metabolites present in high concentrations were relatively influenced less by the disturbances resulting from the different sources of uninduced variation, and were therefore more reliable.

\subsection{Discussion}
The appropriate correlation measures for studied data should closely associate to the prior knowledge of biological correlation. For this study, nonlinear relationship between the abundances of brown plant hopper and white backed plant hopper was detected by a Spearman correlation but not by a Pearson correlation.

However, Pearson correlation coefficient is sensitive to outliers. Biweight midcorrelation is considered to be a good alternative to Pearson correlation since it is more robust to outliers [23Wilcox R: Introduction to Robust Estimation and Hypothesis Testing. Academic Press, San Diego; 1997.].
