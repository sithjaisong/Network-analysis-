\section{Evaluation of correlation methods for co-occurrence network construction of rice crop health survey data}

\subsection*{Introduction}

Network analysis is a promising tool frequently used to describe the pairwise relationships of a large number of variables. Network analysis of pest and disease incidence co-occurrence patterns offers new insight into pest management. Since different correlation measures lead to structurally different co-occurrence network and provide different information, selecting the optimal correlation method is critical. Pests are the blocks in the big wall of yield losses that plant protection specialists must scale due to the need to increase crop productivity. Surveys of farmers’ fields are useful sources of data to help us determine the importance of pests and understand complex relationships within agroecosystems. Our objective is to find an appropriate method to.

%general background information
% using the research ref to support both the background facts and claim for significance
% general problem or the current research focus
% 

\subsection*{Materials and methods}
\textbf{Survey data}

Survey data collected in 450 farmers’ fields in irrigated lowland rice growing areas across South and Southeast Asia Tamil Nadu, India (TMN); Odisha, India (ODS), West Java; Indonesia (WJV); Suphanburi, Thailand (SPB) and Mekong river delta, Vietnam (MKD) were collected from 2013 - 2016. The number of survey were summerize in Table.

\ref{table:Survey_data}

The survey procedure and data were based on a standardized protocol described in ``A survey portfolio to characterize yield-reducing factors in rice'' developed by \citet{Savary_2009_Survey}. Rice injuries were collected in this study including the injuries caused by animal pests, and pathogens, which are harmful to rice plants, and importantly considered to reduce yield productivity. They were evaluated at booting and ripening stage according to survey procedure. Injuries on leaves such as whorl maggot injury (WM), leaffolder injury (LF), bacterial leaf blight (BLB), bacterial leaf streak (BLS), leaf blast (LB), brown spot (BS) were determined as a proportion of injured leaves. Injuries on tillers or hills such as stem rot (SR), sheath rot (SHR), sheath blight (SHB), whitehead (WH), deadheart (DH), silver shoot (SS) caused by gall midges were determined as a proportion of injured tillers or panicles.

Before analysis, data were compacted over time during crop growth. Two types of data were computed, depending on the nature of injury \citet{Savary_2009_Survey}. One is an area under injury progress curve (AUIPC) used for injury variables, which present on the leaves, and for weed infestation. Another is the maximum level at any of the two observations used for injury variables that can be observed on tillers, panicles, and hills, and insect pest count. The area under injury progress curve (AUIPC) \citep{Campbell_1990_Introduction} were calculated by the mid-point method using the following equation: 

\begin{equation}
AUIPC = \sum{\frac{1}{2(X_{i} + X_{i-1})(T_{i} - T_{i-1})}}
\end{equation}

where $X_i$ is percentage (\%) of leaves, tillers or panicles injured due to rice pests (e.g., leaf blast, leaf folder), or number of insects (e.g., plant hoppers, leaf hoppers) per quadrat, or percentage (\%) of weed infestation (ground coverage) at the $i$th observation, $T_i$ is time in rice development stage units (dsu) on a 0 to 100 scale (10: seedling, 20: tillering, 30: stem elongation, 40: booting, 50: heading, 60: flowering, 70: milk, 80: dough, 90: ripening, 100: fully mature) at the $i$th observation and $n$ is total number of observations.


\textbf{Evaluation}

In this study, correlation measures including, Pearson, Spearman, Kendal and Biweight mid correlation were evaluated to discover the true functionally related variables in crop health survey data.  The data will have to follow the assumption of correlation measures. The correlation measures will also be able to effectively capture biological relationships that are well published. I proposed three steps for correlation methods selection: 

\begin{itemize}
\item Testing whether or not the data are normally distributed by visual assessments and statistical tests. I performed Shapiro-Wilk test to check the normality of data distribution.
\item Comparing correlation measures by testing similarity of correlation coefficients. 
\item Identifying the most suitable correlation measure that can capture biological relationships between variables confirm with the published relationships.
\end{itemize}

\subsection{Result}

Evaluation of four correlation measures, including Pearson, Spearman, Kendall correlation, and Biweight mid-correlation shows two groups clustered according to hierarchical clustering using Euclidean distance. The Spearman and Kendall correlation were grouped in rank-base methods, and another group is non-rank-based correlations including Pearson correlation and Biweight mid-correlation. 

When choosing a measure, Spearman correlation was selected because its robustness to noise and to outliers.


Briefly, network analysis revealed that brown spot is the highest connection. Bacterial leaf streak, brown spot, white-backed planthopper, and brown planthopper have high node of connectivity



\textbf{Normality}

To determine normality of the survey data, distribution, skewness, kurtosis, Shapiro-Wick test of each injury were analyzed. The distribution of value of rice injuires were viualzed in histogram

Analysis of the analytical and the uninduced biological standard deviations showed that heteroscedasticity was present both in the analytical error and in the biological uninduced variation (Figure 4A and 4B). In contrast, the relative biological standard deviation, and also the relative analytical standard devi- ation, showed the opposite effect.

Shapiro Wilk $p$-value of each injury indicated that you would reject the null at typical significant levels. That test indicates your data are not normally distributed.


\subsection{Discussion}
The appropriate correlation measures for studied data should closely associate to the prior knowledge of biological correlation. For this study, nonlinear relationship between the abundances of brown plant hopper and white backed plant hopper was detected by a Spearman correlation but not by a Pearson correlation.

However, Pearson correlation coefficient is sensitive to outliers. Biweight midcorrelation is considered to be a good alternative to Pearson correlation since it is more robust to outliers [23Wilcox R: Introduction to Robust Estimation and Hypothesis Testing. Academic Press, San Diego; 1997.].
