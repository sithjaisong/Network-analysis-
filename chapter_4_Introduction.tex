\subsection{Introduction}

%Rice textit{Oryza sativa}) is a major crop in South and Southeast Asian. Generally, rice farmers cultivate 2 rice crops per year, with the typical seasonal crop cycles or rotations being rice-rice-fallow or rice-rice-secondary crops (corn, soybean, peanut). The Food and Agriculture Organization of the United Nations (FAO) estimates that approximately 70 percent of total lowland rice area produces 2 rice crops each year. The first crop is cultivated in the wet season, while another is in the dry season. The important role of seasonal cropping in the temporal dynamics of animal pests and diseases has been studied under farmers field survey in South and Southeast Asia by the use of multivariate techniques (cite savary, and willaet). The previous studies showed that Injuries profiles (the combination of injuries) differ from season to season in term of weather pattern. In the dry season, crop losses were lower than in the wet season. n a previous study based on surveys done in farmers’ rice fields in the region, production situations for lowland rice were shown to be strongly associated with injury profiles, i.e., specific combinations of injuries that affect a rice crop during the course of a cropping season. 

%In the previous chapter, the co-occurrence networks to yield co-occurrence networks, a methodological approach which has already proved fruitful in a variety of different applications with different rice field. In this chapter, I applied the the differential network concept, which attempts attempts to characterize the yield-reducing factors by studying the changes in the the co-occurrence patterns of these networks as opposed to merely analysing the changes in jury data (e.g disease incidence, animal pest injury incidence).
 
%As demonstrated by several studies, differential networks can identify important gene modules implicated in cancer and also provide critical novel biological insights not obtainable using other approaches.
 
The objective of the work reported here was to use a simulation  approach to assess (i) current losses caused by rice pests; (ii)  yield gains, current or expected, associated with certain pest management practices; and (iii) changes in vulnerability due to new  agronomic technologies; to derive research priorities for lowland  rice pest management in tropical Asia.


In this work, we explore the the properties of network of a complex association of rice injuries, which at a systems-level cause a fundamental dynamic rewiring of the injury association network, ultimately differing seasonal expression. Elucidating the rice injuries association represents a key challenge, not only for achieving a deeper understanding of injury association (injury profiles) but also for identifying the unique association. Given that the injury association is governed by a complex network of injuries association, it seems natural to explore network properties which may help elucidate some of different association presenting in the different time frame (seasonal periods).

%==

Using differential network concept defined locally for nodes in the network, we here demonstrate that cancer is characterized by an increase in network entropy. 

We next extend the notion of local entropy to a non-local/global one, i.e for extended subnetworks, and find that non-local entropy measures are less discriminatory of the cancer phenotype. 

We also explore the relation between local differential association, and in the process, elucidate a novel cancer system hallmark. Finally, we discuss the meaning of our results in the context of the entropy-robustness theorem above, and discuss the potential implications of our findings for devising novel cancer therapies with a view to future studies that will attempt to integrate drug sensitivity data with multi-dimensional (mutational, copy-number, epigenetic and transcriptomic) tumour profiles.