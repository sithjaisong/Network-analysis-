\subsection{Introduction}

Rice textit{Oryza sativa}) is a major crop in South and Southeast Asian. Generally, rice farmers cultivate 2 rice crops per year, with the typical seasonal crop cycles or rotations being rice-rice-fallow or rice-rice-secondary crops (corn, soybean, peanut). The Food and Agriculture Organization of the United Nations (FAO) estimates that approximately 70 percent of total lowland rice area produces 2 rice crops each year. The first crop is cultivated in the wet season, while another is in the dry season. The important role of seasonal cropping in the temporal dynamics of animal pests and diseases has been studied under farmers field survey in South and Southeast Asia by the use of multivariate techniques \citet{Savary_2000_Characterization, Willocquet_2008_Simulating}. The previous studies showed that Injuries profiles (the combination of injuries) differ from season to season in term of weather pattern. In the dry season, crop losses were lower than in the wet season. A previous study based on surveys done in farmers’ rice fields in the region of lowland rice were shown to be strongly associated with injury profiles.

In the previous chapter, the co-occurrence networks to yield co-occurrence networks, a methodological approach which has already proved fruitful in a variety of different applications. Basically, plant injuries caused by pests maybe affect on yield production. Therefore in this chapter, I attempted to characterize the yield-reducing factors by studying the changes in the co-occurrence patterns of rice injuries (e.g disease incidence, animal pest injury incidence) at different yield levels.
 
Differential network analysis aims to compare the connectivity of two nodes at 2 different conditions. As demonstrated by several studies, differential networks can identify important nodes implicated in my fields, and also provide critical novel insights not obtainable using other approaches. In this work, I explore the the properties of network of a complex association of rice injuries at different yield levels. Elucidating the rice injuries association represents a key challenge, not only for achieving a deeper understanding of injury association (injury profiles) but also for identifying the unique association. Given that the injury association is governed by a complex network of injuries association, it seems natural to explore network properties which may help elucidate some of different association presenting in the different seasons.

I explored the relation between local differential association and discuss the meaning of the results in the context of the crop health management, and discuss the potential implications of our findings for development of pest management with a view to future studies.




