\section{Introduction}

Pests and diseases to global rice production are significant yield reducing factors. \citet{Oerke_2005_Crop} estimated that rice pests potentially caused losses around 37 percent of global rice production. Additionally, future rice production will need to grow by 2.4 percent per year in order to meet the demands of a growing population \cite{Ray_2013_Yield}. Addressing these yield-reducing factors is essential for food security not only in rice consuming societies, but also for other societies globally.

Rice is predominantly grown in Asia. So much so that thirty--one percent of the rice harvested globally comes from Southeast Asia alone  \citep{oecd_2012_Agriculture}. The highest levels of productivity are found in irrigated areas, the most intensified rice production system. Farmers can grow more than one rice crop per year here. Approximately 45 percent of the rice growing area in Southeast Asia is irrigated, with the largest irrigated areas being found in Indonesia, Vietnam, Philippines and Thailand \citep{Mutert_2002_Developments}. In South Asia, the two major rice-growing countries are India and Bangladesh. India has the largest rice growing area globally, approximately 43 million hectares, and contributes 25 percent of global rice production alone. Combined, rice production in South and Southeast Asia contributes around half of global rice production. If rice production in South and Southeast Asia is threatened, it will significantly affect global rice production. 

Nowadays, developing the strategies of pest and disease management takes into account sustainability, production efficiency, and environmental protection \citep{Mew_2004_Looking}. To achieve this, interactions between pests and human activities must be studied. A survey may provide the necessary data and adequate methods for analyzing survey data can produce preliminary information on their behaviors including major interactions \citep{Savary_1995_Use}. \citet{Savary_2000_Characterization} concluded that the observed injury profiles (\textit{i.e.}, the combination of disease and pest injury that may occur in a given farmer's field) were strongly dependent on production situation.  The authors discussed that pest management strategies should be developed according to the patterns of cropping practices, and production situations. However, interactions among pests, cropping practices, and environments under different locations or over time are difficult to elucidate, which they are important to design the strategies of pest management.

To help visualize and understand these interactions, network analysis seems to provide a promising tool for revealing the interactions among entities within a complex system. It has been applied for many branches of science (\textit{e.g.}, social science, computer science, and biology). A network model is an abstract model composed of a set of nodes or vertices and a set of edges, links or ties connected to the nodes. Nodes usually represent entities and the edges represent their relations. For example, an ecological network of a food web presents nodes as species \citep{krause2003compartments} and edges as ecological relationships, or consider a social network of students in the school present where nodes are students and edges are friendships \citep{moody2001race}.

To design the strategies for sustainable  rice pest control, many studies focused on crop health, which is rather not to focus on a particular aspect of plant protection, crop health must be viewed more holistically. \cite{Savary_1995_Use, Savary_2000_Quantification, Savary_2005_Multiple} discussed that

\begin{itemize}
\item Cropping practices strongly associate the pattern of disease syndromes and damage caused by animal pests;
\item Production situations are strongly related to the occurrence of  individual diseases; and
\item Production situations (a combination of physical and socioeconomic factors that influence agricultural production) represent very large risk factors (positive or negative) for occurrence of disease syndromes. 
\end{itemize}
 
These studies contributed to the body of knowledge needed to address crop health management. We have a good idea of what pests and diseases affect rice, production situation, but we do not always have a clear picture of where individual or groups of pests and diseases occur and how much effect they have on rice yield. One approach that can be used to gain insight into this is to develop network based on a wide range of pest and disease injuries and corresponding yield losses under different rice production situations across Asia. 

\subsection{Objective of the study}

\textbf{The general objectives}
This study aimed to develop network approaches and apply them to analyze crop health survey data conducted in the countries in South and Southeast Asia ( India, Indonesia, Philippines, Thailand, and Vietnam). 

\textbf{The specific objectives}
This research aims 
\begin{enumerate}
\item To develop the network model based on crop health survey data and characterize relationships among components of injuries and cropping practices under the rice agroecosystem in South and Southeast Asia, specifically in India, Indonesia, Vietnam and Thailand.
\item To compare the differential relationships of network models under different seasons or locations (countries,India, Indonesia, Vietnam and Thailand). 
\item To apply network analysis to compare differential interactions in networks model under successive (from low to high) levels of estimated actual yields from India, Indonesia, Vietnam and Thailand.
\end{enumerate}

\subsection{Significance of the study}

This research will be a significant contribution to rice pest management and  new approaches to study the rice crop heath survey data. By understanding the relationships of cropping practices and rice injury profiles which in different location (countries), especially in South and Southeast Asia, this study will be beneficial to the extentionists, agronomists and researchers when they employ effective pest control in their location setting particularly in different location or different environment related to the use of effective rice pest managements.

Once network models based on crop health survey data will be constructed, they will be helpful for plant health authorities and the people who related to crop protection especially for rice. the model will support them to design specific strategies for rice pest and disease management and to limit the impact of these yield reducing factors.

Moreover, the resulting network model will reveal the relationships between farmer practices (cropping practices) with the relationships of co-occurrence patterns of rice pests. The approaches will be assured of a competitive advantage and benefits of quality pest policies provide the recommendations on how to design rice pest management. Importantly, it will also serve as the new approaches for researchers on the subjects of crop losses  and epidemiology.

\subsection{Scope and limitation of the study}

This research is conducted to determine the relationship of farmer's practices and injuries caused by animal pests and diseases.The source of data are contributed from the surveys at the farmer's rice fields in South and Southeast Asia (India, Indonesia, Philippines, Thailand, Vietnam) from 2010 to 2015. To acquire the data conducted by survey based methods, ``survey portfolio'' is applied for all the locations \citep{Savary_2009_Survey}.

%eos