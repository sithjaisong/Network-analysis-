\subsection*{Survey in the field of epidemiological research}\todo{Revise for grammar}

The rice crop system is complex that the number of components is large and their interaction are complex\todo{Revise for grammar}. The dynamic of the system is encompassed by a growing crop, its physical environment, its diseases and pests, and farmers' actions, which influence the whole system. The number of rice pests (diseases, insects, and weeds) to be considered and the levels at which they vary from field to field. A rationale is often deeded to delineate the limits of the system to be addresses\todo{Revise for grammar}.

A survey is a useful tool to provide an adequate account of this diversity and deal with the several pests or variables affecting crop production, which are essential components of systems research in plant protection \citet{Zadoks_1979_Epidem}. Field surveys generate information at the level of the population of crop stands rather than of the individual plot, field, or orchard. Such surveys raise questions regarding variation in injury and its relationships to crop management, crop environment, crop performance, and the occurrence of multiple injuries.


Application of surveys can be found in a huge diversity of research fields, including crop characteristic, cropping practices and single disease or multiple disease and insect. The pest are measurable biological variable that can be used for characterize the injuries profiled (disease and insect injuries syndrome) to stratify farmer's fields account into the characteristic of their production situations.

\cite{Savary_1995_Use} emphasized field surveys as the basic means for acquiring information and expanding knowledge to higher scales. Conversely, field surveys and the associated analytical approaches provide a framework in which to characterize disease intensities and their relationships with attributes (including crop management) of the environments where injuries occur; but they usually do not generate crop loss data \textit{per se}.


The concept of production situation described the set of factors - physical, biological, and socioeconomic - that determine agricultural production\todo{Citation}. This concept is used here within the restricted scope of an individual rice field and from the limited view point of plant protection. A quick overview of the variables listed in the following survey procedure indicates how this concept is operationalized in a survey: this list includes a few key components of rice crop management (which reflect the physical and socioeconomic environments of a crop and their interactions) and a series of rice pests (that is, insects, pathogens, and weeds).

This approach has the considerable advantage of providing quantitative estimate of the contribution of each constraint to the total yield loss. They study may be considered among the first series, where a set of pest constraints of a crop, simultaneously with environmental variables, and variables representing the current cropping practices were considered. Pests only considered some of the ``yield determining variables''. the approaches may also be considerate an attempt to incorporated the concept of production situation \textit{i.e.} the set of environment that lead to a given yield output of the system\todo{Revise for grammar}. 