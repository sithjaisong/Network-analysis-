\section{Literature review}

The world's population is growing rapidly.  It reached 6 billion people in 1999 and is anticipated to reach 8.1 billion in 2025 and 9.6 billion in 2050 \citep{Alexandratos_2012_World}.  Our long-term ability to meet growing needs for food seems uncertain.  Thus, one of the greatest challenges is increasing food production in a sustainable way so that everyone can have adequate food and proper nutrition without over-exploiting the Earth's ecosystems. 

Rice is predominantly produced in Asia, so much so that thirty--one percent of the rice harvested globally comes from Southeast Asia alone \citep{OECD_2012_Agricultural}. The highest levels of productivity are found in irrigated areas, the most intensified rice production systems. Farmers can grow more than one rice crop per year here. Approximately 45 percent of the rice growing country in Southeast Asia is irrigated, with the largest irrigated areas been found in Indonesia, Vietnam, Philippines and Thailand \citep{Mutert_2002_Developments}. In South Asia, the two major rice-growing countries are India and Bangladesh. India has the largest rice growing area globally, about 43 million hectares, and contributes 25 percent of global rice production. Combined, rice production in South and Southeast Asia contributes around half of global rice production. If rice production in South and Southeast Asia is threatened, it will significantly affect global rice production.

Pests in rice production are significant yield reducing factors globally. \cite{Oerke_2005_Crop} estimated that weeds, animal pests, and disease caused losses around 10.2, 15.1 and 12.2 percent of global rice production, respectively. In most Asian countries, rice yields average 3-5 t/ha.  One recent survey estimated that between 120 and 200 million tons of grain are lost yearly to insects, diseases, and weeds in rice fields in tropical Asia \citep{Willocquet_2004_Research}. The mean region-wide rice yield loss due to pests was estimated at 37 percent \citep{Savary_2000_Quantification}.

In crop fields, pests or so-called biotic constraints can be defined as organisms that cause plant injuries and lead potentially to economic losses. Among the pests that attack rice are microorganisms (viruses, mycoplasmas, phytoplasmas, bacteria, oomycetes, and fungi) that can cause diseases, parasitic plants, weeds, invertebrates (insects, mollusks), and even vertebrates such as rats and birds can cause serious damages.

\subsection*{Rice Pests}

Rice fields are human-managed ecosystems, which harbor diverse communities of plant, animal, and microbial species. Many of them indeed benefit with rice plants, such as predators, parasitoids, flowering plants, and soil bacteria \citep{Norton_2010_Rice}. However, there are some species threatening the rice plant's health and can cause the quantity and quality losses. I briefly review some pests including weeds, animal pests, and diseases that are concerned as the important pests and disease in the rice production.

%Within the literature review part, I summarized major rice pests including weeds, animal pests and diseases from several scientifically reliable sources such as websites (\citet{irrirkb}), books (\citet{ouricedisease} and \citet{mew2002handbook}), and articles.

\textbf{Weeds} are plants considered as unwanted plants in the crop growing area and compete for nutrition, water, and light with crops. Weeds can be the cause of severe yield reduction problems. In Asia, they are estimated to cause yield losses up to 80 percent, depending on rice field conditions (crop establishment, water management, field management). Weeds also reduce grain quality, and increase the production costs such as labor cost and input costs \citep{Litsinger_1991_Crop,Savary_2005_Multiple}. The field management and environment are the big factors, which are related weed distribution and degree of infestation in rice fields. For example, in direct seeded rice fields are likely to have high weed pressure than transplanted rice fields \cite{Juraimi_2013_Sustainable}.

\textbf{Rats} cause serious yield reduction problems in many countries in Asia such as Bangladesh, Cambodia, People's Republic of China, India, Indonesia, Lao People's Democratic Republic, Malaysia, Myanmar, Philippines, Thailand, and Vietnam. \textit{Rattus argentiventer} is the major rat pest species in Southeast Asia. Other species are \textit{R. exulans} Peale, \textit{ R. rattus} spp., \textit{R. tanezumi}.  Crop losses due to \textit{R. argentiventer} are typically about 10 to 20 percent. Rat damage in the rice fields is easily observed when a large number of the tillers are cut. They damage to rice plants at all stages from sowing to storage. At the seedling stage, they chop down the young seedlings and also feed on the endosperm. Rat damages can be observed in both the wet and dry seasons, but they increase in rain-fed rice crops in the wet season. Wide bunds in the fields are favorable habitats. They burrow and build the tunnels and nests, which can cause bunds to collapse \citep{Singleton_2003_Impacts}.


\textbf{Golden apple snail} (\textit{Pomacea canaliculata} (Lamarck)) is an exotic species in Asia. Its origin is South America. They were introduced to farmers in countries in Asia to increase their income and as a source of protein in their diet, and also as an aquarium pet \citep{Joshi_2007_Problems}.  Now they are spreading around Asia and have become a major rice pest in the areas where they have founded. Snails attack younger seedlings at deep water levels. \citep{Basilio_1991_Problems} estimated that snail density at 8 snails/square meter could damage the rice seedlings up to 93 percent. However, seedlings become tolerant to snail attacks at the age of 40 days old \citep{Sin_2003_Damage}.  


\paragraph{Insect pests} are serious threats to rice production in nearly all regions, where rice is grown. Yield losses due to insect pests based on surveys from 11 countries were estimated at 18.5 percent \citep{Pathak_1994_Insect}.  The introduction of high yielding technology in the 1960s involving rice varieties with high tillering ability, denser plant spacing, high fertilizer application and irrigation where farmers planted two or three crops per year provided abundant habitat and food sources where the insect pests can reproduce continuously throughout the year. Moreover, the use of nitrogen fertilizers increased the insects' reproductive potential \citep{Bottrell_2012_Resurrecting}. In South and South East Asia, rice is grown in warm, humid environments conducive to the survival and proliferation of key insect pests: stem borers, rice loaders, brown planthoppers, whitebacked planthoppers and green leafhoppers.

\textbf{Stem borers} are ubiquitous throughout rice fields in Asia and cause some damage in every rice field every year. There are six species, but rice yellow stem borer, \textit{Scirpophaga incertulas} (Walker), is the most destructive species in Asia, which may cause 20 percent yield loss in early planted rice crops, and 80 percent in late-planted crops \cite{Ooi_1994_Predators}. The larvae bore into the stems and eat their way down to the base of the plant hollowing out the stem. If the damage occurred during the vegetative stage the central leafs do not unfold, turn browns and die off, this is called ``deadheart''. If this damage occurs during the reproductive stage it results in the drying of the panicles, which may not emerge or do not produce grains. This is called ``whitehead''. Yield losses from stem borer damage can reach up to 95 percent in severe infestations \cite{Pathak_1994_Insect}.

\textbf{Asian rice gall midges}, \textit{Orseolia oryzae} (Wood-Mason) damaged tillers are characterized by pale green color and the stems elongated and converted into hollow tub called ``onion leaf'' or ``silver shoot''. The Asian gall midge could cause significant yield losses of US \$ 80 million annually in India \citep{Bennett_2004_New}. They are found widely in most rice ecosystems; irrigated, rain-fed wetland, upland and deepwater rice area when rice plant is growing at the tillering stage \citep{Pathak_1994_Insect}. 

\textbf{Rice leaf feeders} are diverse such as  rice leaffolder (\textit{Cnaphalocrocis medinalis}), rice whorl maggot (\textit{Hydrellia philippina} Ferinoand), rice hispa (\textit{Dicladispa viridicyanea}) (Kraatz), and  rice thrips \textit{Stenchaetothrips biformis} (Bagnall). They often attack rice in the early rice stages, causing visible injuries on a leaf. \textbf{Rice leaffolder} at the larvae stage fold the leaves by stitching the leaf margins and feed by scraping green leaf tissue Leaves damaged by rice hispa appear irregular translucent white patches paralleling to the leaf veins. \textbf{Rice whorl maggot} at the larvae stage feed on the inner margin of unopened leaves. Damaged leaves show white or transparent patches, pinholes, and are easily broken by the wind. They are commonly found in irrigated fields,but not in upland rice. \textbf{Rice hispa} feeds green tissue on the leaf. The leaves show white streaks parallel to the midrib. The rice hispa is common in rainfed and irrigated wetland environments and is more abundant during the rainy season. \textbf{Rice thrips} are an important pest to rice at young stage (during the seedling stage or two weeks after early sowing). The damage caused by rice thrips can be observed in the dry season than wet season. Leaves damaged by thrips are curl and discolored. However, the injuries caused by these insect pests are not considered important as often does not translate into a yield loss because of plant compensation \citep{Pathak_1994_Insect,Shepard_1995_Rice}.


\textbf{Rice sucking insects} cause enormously economic damages directly by feeding, and indirectly by transmitting many virus diseases. Two species of planthopper, the \textbf{brown planthopper} (BPH), \textit{Nilaparvata lugen}s (Stal); and the \textbf{whitebacked planthopper} (WBPH), \textit{Sogatella furcifera} (Horvath), infest rice. Nymphs and adults feed at the base of the tillers, then plants turn yellow and dry up, turn brown and die. At early stages of heavy infestation round and yellow, and brownish patches appear due to the drying up of the plants. This condition is called ``hopperburn''. It can cover large patches in rice fields. Moreover, the BPHs can transmit rice ragged stunt and rice grassy stunt diseases. Another important planthopper pest of rice is the \textbf{smaller brown planthopper} (SBPH) \textit{Laodelphax striatellus}  \textit{Laodelphax striatellus} Fall \'en, which is mainly found in temperate rice areas. can transmits rice virus diseases such as rice stripe virus disease and rice black-streaked dwarf virus disease \cite{Zhang_2014_Small}. Two species of \textbf{green leafhoppers} (GLH), \textit{Nephotettix malayanus} and \textit{Nephotettix virescens}, infest rice. They feed on the surface of the leaf blades rather than the leaf sheaths and the middle leaves. The rice plants growing in the nitrogen rich fields are vulnerable to GLH.  Plants severely damaged by GHLs showed withering, and complete drying. They can transmit many Rice tungro disease \citep{Ling_1972_Rice}. They also can spread rice tungro disease. \textbf{Rice black bugs} belong to several species, but Malaysian black bug \textit{Scotinophara coarctata} is the most common in Asia. They feed on the rice plant from seedling to maturity growth stages. Their damage can cause reddish brown or yellowing of plants. Leaves also have chlorotic lesions. Severe damage may show the symptom called ``bugburn'' showing wilting of tillers with no visible honeydew deposits or sooty molds. Plants are also stunted; and can develop stunted panicles, no panicles, incompletely excerted panicles, and unfilled spikelets or whiteheads at booting stage \citep{Shepard_1995_Rice}. 

\textbf{Grain sucking insects} causes serious yield losses. \textbf{Rice bugs} or slander rice bugs were found commonly two species; \textit{Leptocorisa oratorius}, and \textit{Leptocorisa acuta} Thunber. They are found commonly in the flowering to milk stages in all rice environments. Nymphs and adults feed on developing grain causing them deformed, spotty and empty grains. They are reported that can cause yield loss up to 30 percent. \textbf{Rice stink bugs} were reported that they damage rice grains, but the most common stink bug is \textit{Nezara viridula} (Linnaeus). They penetrate the grain hull with their strong mouthparts to feed on the endosperm \citep{Shepard_1995_Rice}.

\paragraph{Rice diseases} result in yield reductions of 10 - 15 percent in tropical Asia \citep{Willocquet_2004_Research}. Sheath blight and brown spot are important diseases in rice in Asia, each responsible for 6 percent of yield losses, whereas rice blast, bacterial leaf blight account for 1 - 3 percent and 0.1 percent of yield losses, respectively. Sheath rot, stem rot, and those known as sheath rot complex and grain discoloration are responsible for rice yield losses ranging from 0.1 - 0.5 percent, respectively. All other diseases alone or in combination do not cause more than 0.5 - 1 percent of yield losses based on estimates \citep{Savary_2000_Quantification,Mew_2002_Handbook}. I briefly review some rice diseases that are commonly found in South and South East Asia. 

\paragraph{Foliar diseases} include brown spot, rice blast, narrow brown spot, bacterial leaf blight, bacterial leaf streak, \textit{etc}.

\textbf{Brown spot}, caused by \textit{Cochliobolus miyabeanus} (Ito \& Kuribayashi) Drechs. ex Dastur (anamorph: \textit{Bipolaris oryzae} (Breda de Haan) Shoemaker) is one of the most important disease of rice in Asia, which is reported across South and South-East Asian countries. The disease symptoms can be observed small oval or circular, dark brown spots on leaves or glumes. The disease is seed-transmitted and infected seed is the primary inoculum source. The disease develops quickly in fields with scarce water supply and nutritional imbalance, particularly low nitrogen, phosphorus and potassium. Yield loss caused by this disease have been reported many rice production area \citep{Barnwal_2013_Review}. On average, the lowland rice of tropical and subtropical, \citet{Savary_2006_Quantification} claimed that 10 \% of attainable yield was damaged by this disease.

\textbf{Rice blast}, caused by \textit{Magnaporthe grisea} (Hebert) Barr) \citep{Rossman_1990_Pyricularia} (anamorph: \textit{Pyricularia grisea} Sacc., synonym \textit{P. oryzae} Cavara. This fungal pathogen can overwinter in rice straw and stubble and spreads rapidly by airborne spores. The symptoms can be found on all part of the plant including leaves, leaf collars, necks, panicles, pedicels, roots, and seeds. Leaf blast symptoms are characterized by elliptical or spindle shaped lesion with whitish-gray or greenish center and brown or purple margins with yellow halo. Neck or panicle blast, which is more damaging, appear as a dark necrotic lesion covering partially or complete around the panicle base or secondary branches. It may lead to breaking of panicles resulting in few or no grain setting. Disease is favored by high nitrogen. Soils with poor silica availability are blast conducive. Blast is highly destructive in lowland rice in temperate and subtropical Asia, and upland rice in tropical Asia, Latin America and Africa \citep{Skamnioti_2009_Against,Ou_1985_Rice}.

\textbf{Narrow Brown Spot}, caused by the fungus \textit{Cercospora janseana} (Racib.) O. Const., synonym:\textit{C. oryzae} Miyake (teleomorph: \textit{Sphaerulina oryzina} K. Hara), \todo{Again, teleomorph first, then anamorph} characterized by small, narrow, elongated dark brown spots spreading uniformly over the leaf. Symptoms are usually extensive during the later stages of growth, with lesions appearing just prior to anthesis. Severe damage decreases the grain quality and reduces the milling recovery, but has no significant effect on yield losses \cite{Ou_1985_Rice}. 

\textbf{Bacterial leaf blight} or bacterial blight, caused by \textit{Xanthomonas oryzae} pv. \textit{oryzae} (Ishiyama) Swings \textit{et al.} (synonym: \textit{X. campestris} pv. \textit{oryzae} (Ishiyama) Dye is characterized by yellowing and dry leaves. The disease developed at the tips of the leaf and spread down one or both sides of the leaf and maybe through the middle leaf. The disease occurs in both tropical and temperate environments, particularly in irrigated and rainfed lowland areas. The high wind and rain were believed to aid in rapid dispersal of the pathogen. In severe epidemics, yield losses can range from 20 - 40 percent \citep{Sonti_1998_Bacterial}.

\textbf{Bacterial leaf streak}, caused by \textit{Xanthomonas oryzae} pv. \textit{oryzicola} (Ishiyama) Swings \textit{et al.} is characterized by fine translucent streak between veins. Then the streaks become yellowish-gray, the lesions coalesce, then eventually turn to brown to greyish white causing the leaves to die. This disease is common in South and South-East Asia. Severe infection results in poor grain development, broken rice and deterioration in chemical and nutritional composition \cite{Ou_1985_Rice}.

\paragraph{Diseases on the tiller} include sheath blight, stem rot and bakanae 

\textbf{Sheath blight}, caused by \textit{Thanatephorus cucumeris} (A.B. Frank) Donk (anamorph: \textit{Rhizoctonia solani} K\"uhn) is characterized by irregular lesions usually found on the leaf sheaths (initially water-soaked to greenish gray and later becoming grayish white with a brown margin). Sclerotia (small brown-to-black, rocklike reproductive structures) may be present on the stems. Symptoms are usually observed from tillering to milk stage in a rice crop. This fungus lives in the soil and floats to the top when fields are flooded, contacts rice plants and spreads to adjacent plants. Spread of sheath blight is thus favored by dense crop canopies \cite{Ou_1985_Rice}. It occurs throughout the rice growing areas. Sheath blight causes a yield loss of 6\% across lowland rice fields in tropical Asia \cite{Savary_2006_Quantification}.


\textbf{Stem rot}, which is caused by \textit{Magnaporthe salvinii}, anamorph: \textit{Sclerotium oryzae} Catt, has been reported from most  rice-growing countries. The infected plants showed initially small, irregular black lesions on the outer leaf sheath near water level. The lesions enlarge and reach the inner leaf sheath. The stems became rot causing plants lodge. The small black sclerotia, fungal reproductive can be found in the rotting stem, and carried in stubbles after harvest. The presence of this disease was reported in most of rice growing countries \cite{Ou_1985_Rice}. Reviewed by \citet{Ou_1985_Rice}, this disease was the most common in Vietnam, caused yield loss around 50\%. In the Phillipines, yield loss was estimated range 30 to 50 \% caused by this disease.

\textbf{Bakanae} caused by \textit{Gibberella fujikuroi} Sawada (anamorph: \textit{Fusarium moniliforme} J. Sheld) is commonly found throughout Asia. Infected seedlings elongate abnormally, becomes slender and the leaves turn pale yellow green. Infected plants develop roots at the upper nodes and the whole plant turns yellow. At booting stage, some infected plants die. This disease was reported that it can cause yield loss up to 20 percent in outbreak areas \cite{Ou_1985_Rice}.

\paragraph{Panicle diseases} commonly include sheath rot, dirty panicle complex or grain discoloration, and false smut.

\textbf{Sheath rot}, caused by \textit{Sarocladium oryzae} (Sawada) W. Gams \& D. Hawksworth (Synonym: \textit{Acrocylindrium oryzae} Sawada), reduces grain yield by retarding or aborting panicle emergence, and producing unfilled seeds and sterile panicles. Sheath rot also reduces grain quality by causing panicles to rot and grains to become discolored. The typical sheath rot lesion starts at the uppermost leaf sheath enclosing the young panicles. It appears oblong or as irregular spot with dark reddish, brown margins, and gray center or brownish gray throughout \cite{Ou_1985_Rice}.

\textbf{Dirty panicle} or \textbf{grain discoloration} is characterized by darkening of glumes of spikelets, brown to black, including rotten glumes. This disease caused by multiple fungal species which include, (\textit{Alternaria alternata}, \textit{Altrernaris padwikii}, \textit{Cochilobolus miyabeanus} (synonym: \textit{Bipolaris oryzae}) \textit{Cuvularia} spp., \textit {Fusarium} spp.,\textit{Magnaporthe salvini}, \textit{M. grisea}, \textit{Nigrospora oryzae}, \textit{Sarocladium oryzae}. Theses pathogens contributed to deleterious effect on the quality of seed lots. Rice seedlings are poor at field emergence, and servival. Finally, they are vulnerable to pests and diseases \citep{Ou_1985_Rice} and \citep{Mew_2002_Handbook}.
 
\textbf{False smut} caused by \textit{Villosiclava virens} gen. nov., comb. nov. (anamorph: \textit{Ustilaginoidea virens} (Cooke) Takahashi) is characterized by the grain transformed into a mass of spore ball, which is initially orange, then become olive-green, and then greenish black at maturity \citep{Tanaka_2008_Villosiclava}. False smut of rice is reported in most rice-growing areas of the worlds. In favorable conditions, which are high humidity, and soils with high nitrogen content, yield loss caused by this disease can reach to 7 - 75 percent \citep{Ou_1985_Rice}. Addtionally, the pathogen could produce two kinds of mycotoxins; ustiloxins and ustilaginoidins. They exhibit a variety of biological activities such as  antimitotic activity, cytotoxic activity \cite{Meng_2015_Main}.

\paragraph{Viral diseases} include rice grassy stunt disease, rice ragged stunt disease, and rice tungro disease, \textit{etc}. 

\textbf{Rice grassy stunt disease}, caused by \textit{Rice grassy stunt virus} is transmitted by brown planthopper. Infected rice plants show stunting and proliferation of short, erect, and narrow leaves that are pale green or pale yellow. The virus reduces yields by inhibiting panicle production. Plants can be infected at all growth stages. They are most vulnerable to infection at the tillering stage. Infected stubble and volunteer rice are sources of rice grassy stunt virus. The virus cannot be transmitted via brown planthopper eggs \citep{Ou_1985_Rice,Ling_1972_Rice}.

\textbf{Rice ragged stunt disease}, caused by \textit{Rice ragged stunt virus} is transmitted by brown planthopper. Leaves of infected plants have a ragged appearance. Rice ragged stunt virus infection is particularly high in tropical conditions where rice is planted all year around and provides a continuous host for the brown plant hopper vector. Infected stubble and volunteer rice are sources of rice grassy stunt virus. The disease reduces yield by causing partially exerted panicles, unfilled grains and plant density loss. Infected crops will have significant yield losses of up to 80 percent \citep{Ling_1972_Rice}.

\textbf{Rice tungro disease}, caused by \textit{Rice tungro spherical virus} and \textit{Rice tungro bacilliform virus}, is transmitted by green leafhoppers. Tungro disease can occur during all growth stages of the rice plant. It is most frequently seen during the vegetative phase. Plants are most vulnerable at tillering stage. It causes leaf discoloration, stunted growth, reduced tiller numbers and sterile or partly filled grains. Tungro is one of the most damaging and destructive diseases of rice in South and Southeast Asia. In severe cases, Tungro susceptible varieties infected at an early growth stage can have as high as 100 percent yield loss \citep{Ou_1985_Rice}.

Farmers encountered an estimated average of 37 percent of their yield loss because of pests and disease. This proportion will be reduced when the farmers have efficient pest management. An important concept applied in pest management is ``disease triangle''\todo{Citation}, which explains that an occurrence of plant disease epidemic need three components, virulent pest, susceptible plant and conducive environment. At the farm level, we emphasize the interactions at population level. These interactions depend on not only the physical, biological environment but also man-made activities (Fig.\ref{fig:diseasetriangle}). According to \citet{Savary_2006_Quantification}, pest management tetrahedral first discussed by \cite{Zadoks_1979_Epidem} consists of four elements, a pest, crop, the environment and human. Human is recognized the important role in agroecosystem and should be considered to sustainable pest management \citep{Zadok_1985_Crop}. Humans are not limited to farmers, but also included to farmers' communities, social networks, agro-technology suppliers, food-chain stakeholders, research and extension, and policy-makers. 

Crop, C, incorporates elements pertaining to the host plant genetic make-up, including host plant resistance (HPR), the crop physiology, the crop phenology, and their interactions. C also incorporates microclimate factors that may influence the behaviour of crop-pest systems. This is because, while the microclimate in a crop is driven by physical meso-environment (under E), micro- climate also depends on the crop structure (i.e., its density and architecture). Thus, C not only accounts for the direct effects of HPR, but also for the conditions under which HPR may operate. Furthermore, the expression of resistance depends on the physi- ology of the crop (predisposition; Schoeneweiss, 1975). 

The interactions between human activities and pathogens are considered indirect, but often very strong. For example, crop establishment (e.g., a crop rotation, the choice of a varieties, a crop establishment method), water and nutrient management, and pesticide use are strongly linked to the pest and disease outbreak \cite{Zadoks_1979_Epidem}. This is of importance to design effective pest control management, which is involved the sustainable plant protection (IPM). Moreover,IPM strategies are field-dependent due to the fact that the combination of pests and diseases is vary from different location and crop management practices \cite{Mew_2004_Looking,Savary_2012_Review}. According to \cite{Savary_2000_Quantification,Savary_2000_Characterization}, The amount of yield reduction varies from the different the injury profiles.

\subsection*{Crop establishment and pests}

Crop establishment methods of rice are various, but mainly can be categorized into two types, direct seeding and transplanting. 

Weed composition in the rice field is constantly changing. Some weed species are favorably affected by production practice while some are adversely affected. Direct seeded rice more possibly encounters with weed competition than transplanted rice because they emerge simultaneously with rice seedlings and because of the absence of flooding in the early stages. Generally, weeds such as grasses, sedges, and broadleaf weeds are found in direct seeded rice fields, which dominant weeds are \textit{Echinochloa crus-galli}, and \textit{Leptochloa chinensis} among grasses, \textit{Cyperus difformis}, and \textit{Fimbristylis miliacea} among sedges, and \textit{Ammania baccifera}, \textit{Eclipta prostrata}, and \textit{Sphenoclea zeylanica} in the broadleaf category \cite{Juraimi_2013_Sustainable}.

Compared with transplanted rice, the occurrence of insect pests and diseases is more intense in direct seeded rice because of high plant density and the consequent cooler, more humid, and shadier microclimate inside the canopy \cite{Pandey_2002_Direct}, are conducive to different epidemiological conditions \cite{Willocquet_2000_Effect}. The major insect pests of direct seeded rice are brown planthoppers, stem borers, leaffolders, and gall midges. Important diseases that affect direct seeded rice are blast, ragged stunt disease, yellow orange leaf disease, sheath blight, and dirty panicle \citep{Pongprasert_1995_Insect}. In addition to insect pests and diseases, other pests that attack emerging rice seedlings are the golden apple snail (\textit{Pomacea canaliculata}) and rats, which are more serious problems in direct seeded rice than transplanted rice \cite{Pandey_2002_Direct}.

\subsection*{Nutrient management and pest occurrence}

Nutrition management is one of the most important practices of a high production system, but nutrition management will affect the response of rice to pests, as well as development pattern of pest populations due to the change of conditions \cite{Zadoks_1979_Epidem,Willocquet_2000_Effect}. Soil fertility practices can impact the physiological susceptibility of crop plants to insect pests by either affecting the resistance of individual plant to attack or by weakening plant vulnerability to certain pests. Some studies have also documented how the shift from organic soil management to chemical fertilizers has increased the potential of certain insects and diseases to cause economic losses \cite{Castilla_2003_Interaction}.

Nitrogen, phosphorus and potassium are most often managed by the addition of fertilizers to soils. The others are most often found in sufficient quantities in most soils and no soil amendments are needed to ensure adequate supply unless soil pH limits them. However, I will briefly summarize the effects of nitrogen, phosphorus, and potassium with their relationship to some of the most important rice diseases and insects.

Nitrogen can prolong the vegetative period and increases the proportion of young to mature tissues in rice plants. It can also reduce the amount of cellulose in plant cell walls, predisposing plants to lodging. Aside from these, increased nitrogen was found to reduce phytoalexins in plant cells. Phytoalexins are anti-microbial compounds that build up in plants as a result of infection or stress, and are associated with resistance to fungal and bacterial diseases. On the other hand, if applied correctly, nitrogen allows better plant compensation and tolerance to injury. Bacterial leaf blight can be aggravated by too much nitrogen, and it is even more problematic during the wet season. This is one reason why the recommendation for nitrogen fertilizer during the wet season is lower compared to the recommendation during the dry season. Other than bacterial leaf blight, excessive nitrogen can also be conducive to the development of many other diseases, such as bakanae, bacterial leaf streak, false smut, leaf blast, sheath blight, and sheath rot \cite{Castilla_2003_Interaction}.

As to the effect of nitrogen on the disease, excessive nitrogen application enhanced occurrence of insect pest, especially stem borers, whorl maggots, brown planthoppers and leaffolders \citep{Chau_2003_Impacts,Litsinger_2011_Cultural,Rashid_2014_Effect}. Except whorl maggot and rice thrips, their population did not increase following as the use of nitrogen increase \citep{Chau_2003_Impacts}.

\subsubsection{The effects of phosphorous on pest occurrence}

Just as important as nitrogen, phosphorous is another essential nutrient element that promotes vigorous root development and is responsible for hardening plant tissue. Phosphorus also shortens vegetative period of the plant, as opposed to the effect of nitrogen. Proper timing in the application of phosphorus may lower the incidence of brown spot and leaf blast \cite{Castilla_2003_Interaction}.

In relation to disease incidence, what we know is that phosphorus can reduce the occurrence of bacterial leaf blight and brown spot, but may promote leaf blast and sheath blight of rice. The effect of phosphorus on the population of rice insect pests was not strong \citep{Chau_2003_Impacts,Rashid_2014_Effect}, but the results of \citet{Rashid_2014_Effect} showed that increased phosphorus applicant enhanced the population of brown planthoppers.

\subsubsection{The effects of potassium on pest occurrence}

Among the three essential nutrients mentioned, potassium appeared to be the most consistent and effective in minimizing disease incidence. It was found to lower the infection rate of bacterial leaf blight, leaf blast, and brown spot, sheath blight, sheath rot, stem rot and narrow brown spot \cite{Ou_1985_Rice}. This is because potessium is found to increase the concentration of inhibitory amino acids in the plants as well as phytoalexins \cite{Castilla_2003_Interaction}. Potassium also hardens the plants tissues which can minimize lodging incidence and support faster recovery of injured or stressed plants \cite{Harrewijn_1979}. There were not many studies about the effect of potassium on the rice insect pests. \citet{Rashid_2014_Effect} showed the result in Bangladesh that high potassium application decreased population build up and dry weight of brown plant hoppers and \cite{Salim_2002_Effects} reported that reported that deficiency of potassium in rice plants increased population build up of white backed planthopper and application of high dose of potassium to rice plants decreased population build up of the insect.