\section*{Introduction}
Pests and diseases are significant yield reducing factors in global rice production. \citet{Oerke_2005_Crop} estimated that rice pests potentially caused losses of nearly 37 percent of global rice production. Additionally, human population is continuously increasing every minute. \citet{Ray_2013_Yield} claimed that future rice production will need to grow by 2.4 percent per year in order to meet the demands of a growing population. To maintain food security, we would increase yield productivity, and decrease yield losses. Addressing these yield-reducing factors is essential for food security not only in rice consuming societies, but also for other societies globally.

Rice is predominantly grown in Asia. So much so that thirty--one percent of rice harvested globally comes from Southeast Asia alone \citep{OECD_2012_Agricultural}. The most intensive and productive areas are irrigated. In tropical Asia, with irrigation, farmers can grow more than one rice crop per year. Approximately 45 percent of the rice growing area in Southeast Asia is irrigated, with the largest irrigated areas being found in Indonesia, Vietnam, Philippines, Thailand, etc \citep{Mutert_2002_Developments}. In South Asia, the two major rice-growing countries are India and Bangladesh. India has the largest rice growing area globally, approximately 43 million hectares, and contributes 25 percent of global rice production alone. Combined, rice production in South and Southeast Asia contributes nearly half of global rice production. If rice production in South and Southeast Asia is threatened, it will significantly affect global rice production.  

Nowadays, developing the strategies of pest and disease management takes into account sustainability, production efficiency, and environmental protection \citep{Mew_2004_Looking}. To achieve this, interactions between pests and human activities must be studied. A survey may provide the necessary data and adequate methods for analyzing survey data can produce preliminary information on their behaviors including major interactions \citep{Savary_1995_Use}. \citet{Savary_2000_Characterization} concluded that the observed injury profiles (\textit{i.e.}, the combination of disease and pest injury that may occur in a given farmer's field) were strongly dependent on production situation.  The authors discussed that pest management strategies should be developed according to the patterns of cropping practices, and production situations. However, interactions among pests, cropping practices, and environments under different locations or through time are difficult to elucidate, which they are important to design the strategies of pest management.

To help visualize and understand these interactions, network analysis provides a promising tool for revealing the interactions among entities within a complex system. It has been applied for many branches of science ranging from social science to computer science to biology. A network model is an abstract model composed of a set of nodes or vertices and a set of edges, links or ties connected to the nodes. Nodes usually represent entities and the edges represent their relations. For example, an ecological network of a food web presents nodes as a species \citep{Krause_2003_Compartments} and edges as ecological relationships, or consider a social network of students in the school present where nodes are students and edges are friendships \citep{Moody_2001_Race}.

In order to develop effective strategies of rice pest management, agronomists's studies focused on holistic points of view of crop health, which is not to focus on a particular aspect of plant protection. \citet{Savary_1995_Use,Savary_2000_Quantification} and \citet{Savary_2005_Multiple} discussed that cropping practices strongly associate the pattern of disease syndromes and damage caused by animal pests, production situations are strongly related to the occurrence of individual disease, and production situations (a combination of physical and socioeconomic factors that influence agricultural production) represent very large risk factors (positive or negative) for the occurrences of disease syndromes. 

These studies contributed to the body of knowledge needed to address crop health management. We have a good idea of what pests and diseases affect rice, production situation, but we do not have a clear picture of where individual or groups of pests and diseases occur and how much effect they have on rice yield. One approach that can be used to gain insight into this is to develop networks based on a wide range of pest and disease injuries and corresponding yield losses under different rice production situations across Asia. 

\subsection{Objectives of the study}
This study aimed to develop network approaches and apply them to analyze crop health survey data conducted in the countries in South and Southeast Asia (India, Indonesia, Philippines, Thailand, and Vietnam). 

Specifically, this research aims to:
\begin{enumerate}
\item develop the network model based on crop health survey data and characterize relationships among rice injuries under rice agroecosystems in South and Southeast Asia, specifically in India, Indonesia, Vietnam and Thailand;
\item compare the differential relationships of network models under different seasons or locations; and
\item apply network analysis to compare differential relationships of rice injuries in the network model under successive levels (low, medium and high) of estimated actual yields from different farmers' fields in South and Southeast Asia.
\end{enumerate}

\subsection{Significance of the study}

This research will be a significant contribution to rice pest management and new approaches to study the rice crop heath survey data. By understanding the relationships of cropping practices and rice injury profiles, which in different locations (countries), especially in South and Southeast Asia, this study will be beneficial to the extentionists, agronomists and researchers when they employ effective pest control in their location setting particularly in a different location or different environment related to the use of effective rice pest managements. Once network models based on crop health survey data are constructed, they will be helpful to plant health authorities and the people who related to crop protection especially for rice. The network will support them to design specific strategies for rice pest and disease management and to limit the impact of these yield reducing factors.

Moreover, the resulting network model will reveal the relationships between farmer's practices (cropping practices) with the relationships of co-occurrence patterns of rice pest injuries. The approaches are assured of a competitive advantage and benefits of quality pest policies provide the recommendations on how to design rice pest management. Importantly, it also serve as the new approaches for researchers on the subjects of crop losses  and epidemiology.

\subsection{Scope and limitation of the study}

I determine co-occurrence relationships of rice injuries caused by animal pests and diseases. Data were from  surveys at the farmer's rice fields in South and Southeast Asia (India, Indonesia, Philippines, Thailand, Vietnam) from 2010 to 2013. The survey data were conducted by following ``survey portfolio'' \citep{Savary_2009_Survey}.
