\subsection*{Interactions between coexisting rice pests}


Mixed populations of brown planthoppers and whitebacked planthoppers were reported in many areas. \cite{Naganagoud_2010_Studies} reported the experiment multiple pest yield loss to damage function, and economic injury level Iso-loss equation by mutual regression of incidence of two pests. Leaffolder negatively related to stem borers. \cite{Selvaraj_2012_Determination} report the population of four major rice insect pests (brown planthoppers, whitebacked planthoppers, stem borers, and leaffolders) by light traps in Tungabhadra area, India from 1988 to 1990. The observations revealed that the mean populations of these pests decreased in 1989 and increased in 1990 \todo{Revise para for clarity}.

Many plant diseases are caused by a particular species or even strain in diverse environments. However, there are a number of disease complexes where no one organism causes the full set of symptoms.

Akioshi disease is known as ``hydrogen sulfide toxicity'', causes black crown and root rot, which affect the nutrient absorption. The disease is associated with abnormal soils (silica, potassium, magnesium deficiency). Plants affected by Akiochi become susceptible to \textit{Bipolaris oryzae}, the pathogen of brown spot, which is used as an index of Akiochi \citep{Ou_1985_Rice}.

The dirty panicle (grain discoloration) are complex disease caused my many fungal pathogens on the glumes, kernels or both. The fungi that are reported to be associated with discoloration of grains are \textit{Bipolaris oryzae}, \textit{Alternaria padwickii}, \textit{Pyricularia oryzae}, \textit{Fusarium moniliforme}, \textit{F. graminearum}, \textit{Nigrospora oryzae} and \textit{Curvularia} spp.\cite{Ou_1985_Rice}. However, it is still conflict such as the Pyriculris test the seed from a field heavily infested by blast, but no \textit{P. oryzae} isolated\todo{Revise for grammar and clarity}.

Most plant viruses depend on insect vectors for their survival, transmission and spread. Most of rice virus diseases are transmitted by either plant or leaf hoppers. For example, brown plant hopper can spread the rice gressy stund visus, rice raggged stunt disease,  leaf hopper can spread tungro disease. Hence, population if insect vectors are related to the degree of severity and incedence of virus disease.  \citet{Naganagoud_2010_Studies} claimed that the incidence of tungro depend on vector population and composition (different species of vector differ in transmission ability, availability of virus source, since the virus is nonpersistence, susceptibility and age of host plant, synchronization of these three factor. Moveover, weeds in the rice fields are hosts of virus, and play a vital role to virus disease. tungro with yellow dwarf disease can be found in the Rice plants can be found mix infect


Weed hosts of the vectors and of the virus play a role and climates affect all these factors \cite{Naganagoud_2010_Studies}. Some found the tungro with yellow dwarf disease \cite{Ou_1985_Rice}.  \todo{Revise for grammar}