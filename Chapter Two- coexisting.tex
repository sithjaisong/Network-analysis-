\subsection*{Interactions between coexisting rice pests}

Mixed populations of brown planthoppers and whitebacked planthoppers were reported in many areas. \cite{Naganagoud_2010_Studies} reported the experiment multiple pest yield loss to damage function, and economic injury level Iso-loss equation by mutual regression of incidence of two pests. Leaffolder negatively related to stem borers. \cite{Selvaraj_2012_Determination} report the population of four major rice insect pests (brown planthoppers, whitebacked planthoppers, stem borers, and leaffolders) by light traps in Tungabhadra area, India from 1988 to 1990. The observations revealed that the mean populations of these pests decreased in 1989 and increased in 1990\todo{Revise para for clarity}.

Many plant diseases are caused by a particular species or even strain in diverse environments. However, there are a number of disease complexes where no one organism causes the full set of symptoms.

Akioshi disease is known as ``hydrogen sulfide toxicity'', causes black crown and root rot, which affect the nutrient absorption. The disease is associated with abnormal soils (silica, potassium, magnesium deficiency). Plants affected by Akiochi become susceptible to \textit{Bipolaris oryzae}, the pathogen of brown spot, which is used as an index of Akiochi \citep{Ou_1985_Rice}.

The dirty panicle (grain discoloration) are complex disease caused my many fungal pathogens on the glumes, kernels or both. The fungi that are reported to be associated with discoloration of grains are \textit{Bipolaris oryzae}, \textit{Alternaria padwickii}, \textit{Pyricularia oryzae}, \textit{Fusarium moniliforme}, \textit{F. graminearum}, \textit{Nigrospora oryzae} and \textit{Curvularia} spp.\cite{Ou_1985_Rice}. However, it is still conflict such as the Pyriculris test the seed from a field heavily infested by blast, but no \textit{P. oryzae} isolated\todo{Revise for grammar and clarity}.

The relationships between insect and disease are in the term of vectors\todo{Unclear}. Insects are the vectors of important rice diseases, mostly are viral disease\todo{Revise for grammar}. The incidence of tungro depend on vector population and composition (different species of vector differ in transmission ability, availability of virus source, since the virus is nonpersistence, susceptibility and age of host plant, synchronization of these three factor\todo{Revise for grammar}. Weed hosts of the vectors and of the virus play a role and climates affect all these factors \cite{Naganagoud_2010_Studies}. Some found the tungro with yellow dwarf disease \cite{Ou_1985_Rice}\todo{Revise for grammar}

\subsection*{Survey in the field of epidemiological research}\todo{Revise for grammar}

The rice crop system is complex that the number of components is large and their interaction are complex\todo{Revise for grammar}. The dynamic of the system is encompassed by a growing crop, its physical environment, its diseases and pests, and farmers' actions, which influence the whole system. The number of rice pests (diseases, insects, and weeds) to be considered and the levels at which they vary from field to field. A rationale is often deeded to delineate the limits of the system to be addresses\todo{Revise for grammar}.

A survey is a useful tool to provide an adequate account of this diversity and deal with the several pests or variables affecting crop production, which are essential components of systems research in plant protection \citet{Zadoks_1979_Epidem}. Field surveys generate information at the level of the population of crop stands rather than of the individual plot, field, or orchard. Such surveys raise questions regarding variation in injury and its relationships to crop management, crop environment, crop performance, and the occurrence of multiple injuries.


Application of surveys can be found in a huge diversity of research fields, including crop characteristic, cropping practices and single disease or multiple disease and insect. The pest are measurable biological variable that can be used for characterize the injuries profiled (disease and insect injuries syndrome) to stratify farmer's fields account into the characteristic of their production situations.

\cite{Savary_1995_Use} emphasized field surveys as the basic means for acquiring information and expanding knowledge to higher scales. Conversely, field surveys and the associated analytical approaches provide a framework in which to characterize disease intensities and their relationships with attributes (including crop management) of the environments where injuries occur; but they usually do not generate crop loss data \textit{per se}.


The concept of production situation described the set of factors - physical, biological, and socioeconomic - that determine agricultural production\todo{Citation}. This concept is used here within the restricted scope of an individual rice field and from the limited view point of plant protection. A quick overview of the variables listed in the following survey procedure indicates how this concept is operationalized in a survey: this list includes a few key components of rice crop management (which reflect the physical and socioeconomic environments of a crop and their interactions) and a series of rice pests (that is, insects, pathogens, and weeds).

This approach has the considerable advantage of providing quantitative estimate of the contribution of each constraint to the total yield loss. They study may be considered among the first series, where a set of pest constraints of a crop, simultaneously with environmental variables, and variables representing the current cropping practices were considered. Pests only considered some of the ``yield determining variables''. the approaches may also be considerate an attempt to incorporated the concept of production situation \textit{i.e.} the set of environment that lead to a given yield output of the system\todo{Revise for grammar}. 


\subsection*{Common techniques for analysis of survey data}

Methodologies used to interpret the survey data are developed for studies of the relationships for injuries - yield losses. The development of theses studied started focusing on single-disease (pest) pathosystem to more complex. The classic analytical approaches aim to assess group-wise differences, either in a univariate \textit{i.e.} parameter-by-parameter fashion or using multivariate techniques (\textit{e.g.} multivariate analysis of variance (MANOVA), correspondence analysis (CA), multiple correspondence analysis (MCA), or principle component analysis (PCA)). Univariate methodologies are frequently used to reduce a possibly large number of measured analyses to only those that show the strongest response under the investigated conditions. Examples for such univariate approaches are significance testing methods and simple linear regression model. However, univariate methods fail to apply to different production situation or the scope that model can be applied.


Therefore, multivariate analysis methods seek to capture not only changes of a single pest. Probably the most prominent multivariate analysis techniques applied in the field of survey data are principal component analysis, cluster analysis and correspondence analysis. These analyses do not always; and to yield loss estimate\todo{Revise for clarity}. They can inform us the close relationship between changing production situations and shift crop health syndromes (changes in the dynamics of group disease or crop-limiting factors in an area)\todo{Revise for clarity}. 

Farmers' fields survey datasets are of immense value to plant pathologists; they, however, are generally heterogeneous in format (being qualitative or quantitative) and/or precision. A number of methods, some now old and seldom used, and other more recent and comparatively easier to use, are available to process this type of information. The approached used to analyst these data included three main steps: cluster analysis, multiple correspondence analysis and logistic regression.


Cluster analysis represents another unsupervised multivariate technique suitable for the analysis of survey data with hierarchical cluster analysis and $k$-means clustering being the most prominent representatives. In general, clustering methods group and visualize samples according to intrinsic similarities in their measurements, irrespective of sample groupings. 