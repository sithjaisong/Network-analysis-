\subsubsection{Key factor analysis}

Identification key factors in a network is very useful and widely used in social science. The way to identify key factors is to compare relative values of centrality such as eigenvector centrality and betweenness. Its apparent that many measures of centrality are correlated \cite{Valente_2008}.  The residual of linear relationship between eigenvector centrality and betweeness and regress betweeness on eigenvector centrality can used as an indicators of key factors. A node or factors with higher levels of betweenness and lower eigenvector centrality can be inferred that is central to the functioning of the network. Nodes with lower levels of betweeness and higher eigenvector centrality may be inferred that they are key to the functioning of the network.