\subsection*{Summary}

This literature review presented a brief introduction of network analysis and concise concepts and methods\todo{what about common pests and diseases of tropical and sub-tropical rice in Asia? I think you should wrap that up here too}. Briefly, a network is usually represented by sets of nodes (or vertices) connected by edges (or links) in various ways. Networks can be categorized to four types, social network, information, technology network, and biological network. Even though four types of networks are described and applied in different context, they share a common empirical focus on relational structure and a similar set of mathematical analyses. Network models are cable of presenting unique values, which traditional approaches cannot present. Network analysis was discussed as applied to two broad areas of study of plant diseases. It firstly was applied to study plant disease epidemics. For example, networks of \textit{P. ramorum} spread through plant nurseries trade. The results enabled us to understand the directions and processes of disease spread. Additionally, they could predict the movement of disease flows, and improve the implementation of plant disease policy. Secondly, molecular plant pathology showed two applications of network applications. The first use is to apply networks to model large and complex biological datasets. Another use is the consideration network structure to understand biological system. Emergent properties of network structure influences may be identified, measured and analyzed to yield better explanations of the experiments being observed. While numbers of documented plant pathological studies using network analysis are sparse, the literature presented in this review showed a clear and compelling case for plant pathologists to expand the understanding of and use network concepts and methods. 
Biological networks are context-specific and dynamic in nature. Under different conditions, the topology of the network changes. Network approaches (differential network, dynamic network analysis) are successful to yield insight on biological system under different conditions or dynamic processes. Network analysis concepts and methods augment existing approaches and provide tools for exploring complex relationships, which have been widely acknowledged as influential but difficult to measure using traditional methods. 