\sebsection{Introduction}
Rice textit{Oryza sativa}) is a major crop in South and Southeast Asian. Generally, rice farmers cultivate 2 rice crops per year, with the typical seasonal crop cycles or rotations being rice-rice-fallow or rice-rice-secondary crops (corn, soybean, peanut). The Food and Agriculture Organization of the United Nations (FAO) estimates that approximately 70 percent of total lowland rice area produces 2 rice crops each year. The first crop is cultivated in the wet season, while another is in the dry season. The important role of seasonal cropping in the temporal dynamics of animal pests and diseases has been studied under farmers field survey in South and Southeast Asia by the use of multivariate techniques \citet{Savary_2000_Characterization, Willocquet_2008_Simulating}. The previous studies showed that Injuries profiles (the combination of injuries) differ from season to season in term of weather pattern. In the dry season, crop losses were lower than in the wet season. A previous study based on surveys done in farmers’ rice fields in the region of lowland rice were shown to be strongly associated with injury profiles.
In the previous chapter, the co-occurrence networks to yield co-occurrence networks, a methodological approach which has already proved fruitful in a variety of different applications.  Plant injuries caused by pests maybe affect yield production. Therefore, in this chapter, I attempted to characterize the patterns of rice injuries by studying the changes in the co-occurrence patterns of rice injuries (textit{e.g} disease incidence, animal pest injury incidence) at different yield levels.
Differential network analysis aims to compare the connectivity of two nodes at 2 different conditions. As demonstrated by several studies, differential networks can identify important nodes implicated in my fields, and also provide critical novel insights not obtainable using other approaches. In this work, I explore the the properties of network of a complex association of rice injuries at different yield levels. Elucidating the rice injuries association represents a key challenge, not only for achieving a deeper understanding of injury association (injury profiles) but also for identifying the unique association. Given that the injury association is governed by a complex network of injuries association, it seems natural to explore network properties which may help elucidate some of different association presenting in the different seasons.
My overall objective was to explored the relation between local differential association and discuss the meaning of the results in the context of the crop health management, and discuss the potential implications of our findings for development of pest management with a view to future studies.


evaluate the impact of adapting the SimCast model to predict the risk of late blight based on daily and monthly weather data. My first objective for this project was to develop disease prediction models based on daily and monthly weather means and compare to results based on hourly weather data. The second objective was to compare the blight unit predictions of models constructed from weather data sets specific to potato growing regions with models constructed with a data set that represents a broad range climate types. The third objective was to compare late blight risk predictions based on hourly, daily, and monthly weather averages to observed late blight severity data sets from four countries. 

\subsection{Materials and Methods}

The first objective was to develop ____. TO do this, I compared _____
The second objective was to cokpare the 
The third objective was to compare SimCast Daily Means and SimCast Monthly Means output to disease severity observations from several countries. For the third objective, late blight severity and hourly weather data from 19 cultivar, site-year combinations 

I used survey data were collected from farmers' fields ,analyzed co-occurrence relationships of rice injuries, and performed network analysis as described previously. Actual yield estimates were collected from each farmers' fields surveyed. Before analyzing. Three yield levels were  I grouped the survey data into 6 groups by different yield levels and seasons.

\textbf{Network Comparison Measures}

For the pair of $x_{i}$ injury and $y_{i}$ variable, I denoted the correlation coefficient based on Spearman's correlation coefficient by $c_{xy}^1$ and $c_{xy}^2$ in networks 1 and 2, respectively. To test whether the 2 correlation coefficients were significantly different of each network, I used Spearman's correlation coefficients, the p-values of the correlation test, the difference of the 2 correlations, the corresponding $p$-values, and the result of Fisher's z-test. First, I applied the Fisher $z$-transformation in order to stabilize variances due to sample size.

\begin{equation}
z_{xy} = \frac{1}{2} \log\left[{\frac{1 + c_{xy}}{1 - c_{xy}}}\right]
\end{equation}

if we let $z_{xy}^D $ and $z_{xy}^W$ denote the $z$- transformation for dry and wet season variable pairs, respectively.

Next, differences between the two correlations can be test using following. 

\begin{equation}
\triangle z_{xy} = \frac{z_{xy}^D - z_{xy}^W}{\sqrt{\frac{1}{N_{D}-3}+ \frac{1}{N_{W}-3}}}
\end{equation}

$N_{D}$ and $N_{W}$ represent the sample size for each of season network for each country. The $Z$ has an approximately Gaussian distribution under null hypotheses that the population correlations are equal. The pair-wise correlation significants are considered at $p$vlaue < 0.05.

