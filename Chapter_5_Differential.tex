\chapter{DIFFERENTIAL NETWORKS REVEAL THE DYNAMICS OF ANIMAL INJURIES AND DISEASE CO-OCCURRENCE DIFFERENTIAL NETWORKS}

\subsection{INTRODUCTION}

Rice (\textit{Oryza sativa}) is the most important human food crop in the world, feeding more an extraordinarily high portion of the total planted area in South, Southeast Asia. The Food and Agriculture Organization of the United Nations (FAO) estimates that approximately 70 percent of total lowland rice area produces two rice crops each year, main crop in the wet season, while another is in the dry season. The important role of seasonal cropping in the temporal dynamics of animal pests and diseases has been studied under farmers field survey in South and Southeast Asia by the use of multivariate techniques \citet{Savary_2000_Characterization, Willocquet_2008_Simulating}. These studies showed that injuries profiles (the combination of injuries) differ from season to season due to weather patterns, which in dry season, crop losses were lower than in the wet season because of lower incidence and severity of pests and diseases \citep{Litsinger_1991_Crop}. Additional a previous study based on surveys done in farmers’ rice fields in the region of lowland rice were shown that injury profiles were strongly associated with season.

In the previous chapter, the co-occurrence networks showed  co-occurrence networks, a methodological approach which has already proved fruitful in a variety of different applications.  Plant injuries caused by pests maybe affect yield production. Therefore, in this chapter, I attempted to characterize the patterns of rice injuries by studying the changes in the co-occurrence patterns of rice injuries (\textit{e.g} disease incidence, animal pest injury incidence) in different season and different yield levels.

Differential network analysis aims to compare the connectivity of two nodes at two different conditions. As demonstrated by several studies, differential networks can identify important nodes implicated in my fields, and also provide critical novel insights not obtainable using other approaches. In this work, I explore the the properties of network of a complex association of rice injuries at different yield levels. Elucidating the rice injuries association represents a key challenge, not only for achieving a deeper understanding of injury association (injury profiles) but also for identifying the unique association. Given that the injury association is governed by a complex network of injuries association, it seems natural to explore network properties which may help elucidate some of different association presenting in the different seasons.

In this chapter, I employ a differential network topology method to examine the co-occurrence relationships of rice injuries from survey data. I use graph theory methods to examine the topological feature dynamic of a co-occurrence network corresponding to different seasons, and production environments. The co-occurrence networks were built from differentially co-occurring injuries. I extract significantly differential co-occurring injuries from co-occurrence networks, which represent different seasons, to identify which injuries that may be involved specifically curtain season. I postulate that these selected injuries may contribute to the difference in the co-occurrence patterns in different season. Furthermore, I identified the injuries associated with yield from networks at different yield levels. Finally, I suggest key injuries that may contribute to yield reduction under a curtain production environment. The goal is to leverage insights to better understand the rice injury co-occurrence that may contribute to pest management development.

\newpage
\subsection{MATERIALS AND METHODS}

\textbf{Differential co-occurrence network construction}

The survey data were pre-processed by using methods described in the previous chapter. Subsequently, I applied the method proposed by \citet{Fukushima_2013_Diffcorr} to identify differentially co-occurrence links. The difference of co-occurrence of injury $x$ and $y$ between two conditions ($A$ and $B$) was quantified by Fisher's $z$-test.
 
For the pair of $x$ injury and $y$ injury, I denoted the correlation coefficient based on Spearman's correlation coefficient by $r_{xy}^A$ and $r_{xy}^A$ in networks of condition $A$ and condition $B$, respectively. To test whether the 2 correlation coefficients were significantly different, correlation coefficients for each of the 2 conditions, $r_{xy}^A$ and $r_{xy}^B$, were transformed into $Z_{xy}^A$ and $Z_{xy}^B$, respectively.


The Fisher's transformation of coefficient $r_{xy}^A$ is defined by
\begin{equation}
\label{eq:zvalue}
Z_{xy} = \frac{1}{2} \log\left[{\frac{1 + r_{xy}}{1 - r_{xy}}}\right]
\end{equation}

Next, The $p$-value of the difference in $Z_{xy}$ values was calculated using the standard normal distribution.

\begin{equation}
\label{eq:pofz}
p(Z\geq \left | \frac{Z_{xy}^A - Z_{xy}^B}{\sqrt{\frac{1}{N_{A}-3}+ \frac{1}{N_{A}-3}}} \right |
\end{equation}

Next, The $p$-value of the difference in Z values was calculated using the standard normal distribution

$N_{A}$ and $N_{A}$ represent the sample size for each of condition. The $Z$ has an approximately Gaussian distribution under null hypotheses that the population correlations are equal. The pairwise correlation significants are considered at $p$-value < 0.05.


\textbf{Differential co-occurrence network in different seasons}

Consider any two injuries $x$ and $y$ in the survey data, let $r_{xy}^D$ and $r_{xy}^W$ be the Spearman’s correlation coefficient calculated separately over the samples in dry and wet, respectively. I constructed differential co-occurrence networks that are specified by adjacency matrix $A^{diff}$ = $(A_{xy}^{diff})$ where the entry $A_{xy}^{diff}$ quantified by following:   


\begin{equation}
A_{xy}^{diff} = \left\{\begin{matrix}
 1 & \text{when } r_{xy}^D > r_{xy}^W \text{ at } P_{z_{xy}} \text{-value}  < 0.05  \\ 
 0 & \text{when } P_{z_{xy}}  \text{-value}  > 0.05                              \\ -1 & \text{when } r_{xy}^W > r_{xy}^D \text{ at } P_{z_{xy}} \text{-value}  < 0.05 \end{matrix}\right.\end{equation}


For this differential co-occurrence network,$A_{xy}^{diff}$ equals 1 depending on whether any injury pairs show significantly higher correlation coefficient of co-occurrence  in dry season than wet season, but -1 is vice versa, and if it equal 0, meaning that co-occurrence level of injury pairs were not different in dry and wet season. \ref{fig:pipeline3} illustrated the differential co-occurrence network at different seasons.

\begin{figure}[h]
\centering
\includegraphics[width = 1\textwidth]{figures/pipeline3.pdf}
\caption[Differential analysis of crop health survey data in season]{Schematic showing differential analysis in seasons. Co-occurrence networks are measured in each of two seasons (left) resulting in interactions (black). Dry season is subtracted from wet season to create a differential co-occurrence network (right), in which the significant differential interactions are those that positive (red) or negative (blue) in score after the shift in conditions, which means differential in dry, and wet season, respectively.}
\label{fig:pipeline3}
\end{figure} 

\textbf{Difference of co-occurrence network of rice injuries at different yield levels}

Consider any two injuries $x$ and $y$ in the survey data, let $r_{xy}^L$ and $r_{xy}^H$ be the Spearman’s correlation coefficient calculated separately over the samples in $L$ and $H$ yield level, respectively. I constructed differential co-occurrence networks that are specified by adjacency matrix $A^{diff}$ = $(A_{xy}^{diff})$ where the entry $A_{xy}^{diff}$ quantified by following:   

\begin{equation}
A_{xy}^{diff} = \left\{\begin{matrix} 1 & \text{when } r_{xy}^L > r_{xy}^H \text{ at } P_{z_{xy}} \text{-value} < 0.05  \\  0 & \text{otherwise}                             
\end{matrix}\right.
\end{equation}

For  this differential co-occurrence network,$A_{xy}^{diff}$ equals 1 depending on whether any injury pairs show significantly higher co-occurrence level in low yield level than high yield state, and if it equal 0, meaning that co-occurrence level of injury pairs were not or lower different in low yield level state. \ref{fig:pipeline4} illustrated the differential co-occurrence network at different seasons.

\begin{figure}[h]
\centering
\includegraphics[width = 1\textwidth]{figures/pipeline4.pdf}
\caption[Differential analysis of crop health survey data at different yield levels]{Schematic showing differential analysis at different yield levels. Co-occurrence networks are measured in each of two different yield levels (left) resulting in interactions (blue). The network at low yield level network is subtracted from high yield level to create a differential co-occurrence network (right), in which the significant differential interactions are those that positive in score after the shift from low to high yield state.}
\label{fig:pipeline4}
\end{figure} 

\textbf{Topological properties}
To investigate the structural properties of differential networks, I calculated topological features for each node in the network with the \textbf{igraph} package. This feature set included node degree, clustering coefficient, and betweenness. 


\clearpage
\subsection{RESULTS}

Differential network approach to constructing response networks enables easy comparison of the generated graphs. This, in turn allowed identification of the differences between the responses of co-occurrence relationships of rice injuries. In particular, in the differential network, I identified injuries and network components that are relevant for responsive condition as well as responsible for condition. 


\subsubsection{Construction of differential co-occurrence networks of rice injuries at different seasons}

I determined differential co-occurrence patterns of rice injuries of survey data in dry and wet season. The Differential co-occurrence network in season (DCON-S), presenting pairs of injuries (nodes) connected with significantly different co-occurrence relationships (edges), are showed in Figure. \ref{fig:difseasonnetwork_CP} to \ref{fig:difseasonnetwork_WJ}. DCON-S at Central Plain (Figure. \ref{fig:difseasonnetwork_CP}) reveals SHB, SHR, and RS showing significantly different co-occurrence in both dry and wet season. They are likely to observed theses injuries. DP, SHB are high-betweenness that have both differential links passed through. This indicates that SHB can present in both dry and wet season, and can co-occur with injuries such as HB and WH in wet season. DP is interring node because it is high betweenness and node degree. DP shows differential co-occurrence only in dry season. Obviously, DP are more likely to be observed in dry than wet season because there are many differential links to increase it. DCON-S at Odisha \ref{fig:difseasonnetwork_OD} reveals three groups of injuries shared differential co-occurrences.  WM, BS, and SHB present differential links in dry season. These injuries and their associations may be observed higher in dry than wet season. RH, FS, SS, and DH only show differential relationships in wet season, and they have higher chances to co-occur in wet than dry season. LB different both in dry and wet season. It can co-occur with injuries in both season. LB may present in dry season, if HB occur, and if LM occur in wet season, LB may be found in wet season as well.
DCONS at Red River Delta \ref{fig:difseasonnetwork_RR} reveals that DP, BS, RTH, and LF show differential links in dry and wet season. They have different differential co-occurrence patterns. For example, DP may highly occur in wet season, if WH highly is observed in wet season, but in dry season, if BS is highly observed. DCON-S (Figure. \ref{fig:difseasonnetwork_TM}) at Tamil Nadu shows two injury pairs express significantly co-occurrence in dry season, which are WH-BS, and DP-RTH, and one injury pair (SHB-LF) apparently occur together in wet season. DCON-S at West Java (Figure.\ref{fig:difseasonnetwork_WJ} revealed that BS, NBS, SHB, and BLS form differential co-occurrence in both seasons. They can occur in dry and wet season, and form co-occurrence with some injuries. BLB, DP, LF occurred in dry season, then those injuries may occur. WM, DH, RH presented in wet season, then those injuries could be observed highly in wet season as well.

\begin{figure}
\centering
\includegraphics[width = 1\textwidth]{figures/difseasonCP/difseasonCP.pdf}
\caption{Differential co-occurrence network of rice injuries in different seasons at Central Plain, Thailand}
\label{fig:difseasonnetwork_CP}
\end{figure} 

\begin{figure}
\centering
\includegraphics[width = 1\textwidth]{figures/difseasonOD/difseasonOD.pdf}
\caption{Differential co-occurrence network of rice injuries in different seasons at Odisha, India }
\label{fig:difseasonnetwork_OR}
\end{figure}


\begin{figure}
\centering
\includegraphics[width = 1\textwidth]{figures/difseasonRR/difseasonRR.pdf}
\caption{Differential co-occurrence network of rice injuries in different seasons at Red River Delta, Vietnam}
\label{fig:difseasonnetwork_RR}
\end{figure}


\begin{figure}
\centering
\includegraphics[width = 1\textwidth]{figures/difseasonTM/difseasonTM.pdf}
\caption{Differential co-occurrence network of rice injuries in different seasons at Tamil Nadu, India}
\label{fig:difseasonnetwork_TM}
\end{figure}


\begin{figure}
\centering
\includegraphics[width = 1\textwidth]{figures/difseasonWJ/difseasonWJ.pdf}
\caption{Differential co-occurrence network of rice injuries in different seasons at West Java, Indonesia}
\label{fig:difseasonnetwork_WJ}
\end{figure}

%================================
\subsubsection{Differential co-occurrence networks of rice injuries at yield levels}

Comparison of the injury co-occurrence networks differed by season and the generic stress response network induced by different yield levels revealed features unique to each graph. In particular, I not only detected the presence of injuries, but I was able to see how they interacted with other genes in their respective networks. 
The injuries showed in the differential networks are the injuries showed co-occurrence patterns represented by edges. Edges were determined as co-occurrence correlation that express significantly higher in lower yield level. 

In this study, three successive yield classes were defined, in order to enable a better description of actual yield, from low (< 4 ton/ha), medium (4 – 6 ton/ha),  high (> 6 ton/ha) yield levels. Figure \ref{fig:yield_level_bar} shows the number of farmers’ fields surveyed classified in each season, and production environment.

\citet{Berry_2014_Deciphering} recommended that a co-occurrence network will be more reliable, it should be produced using a minimum of 25 samples or observations. From figure \ref{fig:yield_level_bar}, to be able compare the networks at different yield levels, I chose the data set, which are medium and high yield level of Central Plain, low and medium yield level of Odisha, medium and high yield level at Red River Delta, low and medium yield level of Tamil Nadu, and medium and high yield level of West Java. 

The resulted networks, differential co-occurrence network in yield (DCON-Y), depicts the associations of injury pairs significantly expressing in lower yield state but absent in higher yield state. Figures.\ref{fig:difyieldnetwork_CP} presents DCON-Y between medium and high yield level. It comprised of two groups of related injuries. Form node properties, DP, SHB, BS, RH, FS, and DH are high node degree and betweenness, which are likely to occur in this state because there are many connection, which can increase other injuries related, be induced by them as well. BS, SHB, are DP recorded in survey data showed these injuries were observed higher in medium than high yield data set(Fiure. \ref{fig:nodepropdifyield_CP}). DCON-Y compared between low yield and medium yield level at Odisha (Figure \ref{fig:difyieldnetwork_OD}) shows closely associated injures (DH, RH, FS, BS, WH). Their relationships were captured only in low yield state, and WH, BS, and FS are injuries connecting to all the injuries. Survey data also found that these injuries were observed in high frequency at high level of incidence (Figure \ref{fig:nodepropdifyield_OD}). In Red River delta, DCON-Y (Figure.\ref{fig:difyieldnetwork_RR}) reveals three injury combinations (DP-WH, GS-BB, and WM-LB-RTH) presenting in medium yield level, not in high yield level. So their co-occurrences may affect to yield losses between medium to high yield levels. Based on the survey data, DP, WM, and WH were found more incidence in low yield than high yield level (Figure. \ref{fig:nodepropdifyield_RR}). Apparently, in Tamil Nadu, Figure.\ref{fig:difyieldnetwork_TM} showed the co-occurrence pattern between SHB and WH that was found only in low yield level state, even though, from survey data, WH were found higher incidence in medium yield level than low yield level, but SHB were observe high incidence in low yield level than medium yield level (Figure \ref{nodepropdifyieldnetwork_TM}). In West Java, DCONY (Figure. \ref{fig:difyieldnetwork_TM}) reveals two groups of associated injuries that significantly express in medium yield level state. Form topological features of this DCON-Y, FS, RS, SHB, and LB are high betweenness nodes, which they are more likely to occur or be observed in this state because there are many pathway to increase them. SHB based on survey data also were highly found in medium than high yield level (Figure.\ref{fig:nodepropdifyield_WJ}).  

\begin{figure}
    \centering
        \includegraphics[width = 1\textwidth]{figures/yield_level_bar/yield_level_bar.pdf}
\caption{Bar graphs showing number of farmers' fields classified by different yield levels in each production environment.}
\label{fig:yield_level_bar}
\end{figure}

\begin{figure}
    \centering
    \begin{subfigure}[b]{1\textwidth}
        \includegraphics[width = 0.9\textwidth]{figures/difyieldCP/difyieldCP.pdf}
        \caption[Differential co-occurrence network of rice injuries in different yield levels at Central Plain, Thailand]{Differential co-occurrence network of rice injuries in different yield levels at Central Plain, Thailand. The layout of the network graph is based on the Fruchterman-Reingold algorithm, which places nodes with stronger or more connections closer to each other.}
        \label{fig:difyieldnetwork_CP}
    \end{subfigure}
    \begin{subfigure}[b]{0.9\textwidth}
        \includegraphics[width = 1\textwidth]{figures/yield_dif_nodepropCentral_Plain/yield_dif_nodepropCentral_Plain.pdf}
        \caption[Three centrality measures of the nodes in co-occurrence network of rice injuries in dry season at Central Plain.]{Three centrality measures of the nodes in co-occurrence network of rice injuries in dry season at Central Plain. A: node degree, B:clustering coefficient, and C:Betweenness.}
        \label{fig:nodepropdifyield_CP}
    \end{subfigure}
    \caption{Differential network analysis of survey data in different yield levels at Central Plain, Thailand.}
    \label{fig:difyield_CP}
\end{figure}
 
\begin{figure}
    \centering
    \begin{subfigure}[b]{1\textwidth}
        \includegraphics[width = 1\textwidth]{figures/difyieldOD/difyieldOD.pdf}
        \caption[Differential co-occurrence network of rice injuries in different yield levels at Odisha, India.]{Differential co-occurrence network of rice injuries in different yield levels at Odisha, India. The layout of the network graph is based on the Fruchterman-Reingold algorithm, which places nodes with stronger or more connections closer to each other.}
        \label{fig:difyieldnetwork_OD}
    \end{subfigure}
    \begin{subfigure}[b]{1\textwidth}
        \includegraphics[width = 1\textwidth]{figures/yield_dif_nodepropOdisha/yield_dif_nodepropOdisha.pdf}
        \caption[Three centrality measures of the nodes in co-occurrence network of rice injuries in dry season at Odisha, India.]{Three centrality measures of the nodes in co-occurrence network of rice injuries in dry season at Odisha, India. A: node degree, B:clustering coefficient, and C:Betweenness.}
        \label{fig:nodepropdifyield_OD}
    \end{subfigure}
    \caption{Differential network analysis of survey data in different yield levels at Odisha, India.}
    \label{fig:difyield_OD}
\end{figure}
 
 \begin{figure}
    \centering
    \begin{subfigure}[b]{1\textwidth}
        \includegraphics[width = 1\textwidth]{figures/difyieldRR/difyieldRR.pdf}
        \caption[Differential co-occurrence network of rice injuries in different yield levels at Red River Delta, Vietnam. ]{Differential co-occurrence network of rice injuries in different yield levels at Red River Delta, Vietnam. The layout of the network graph is based on the Fruchterman-Reingold algorithm, which places nodes with stronger or more connections closer to each other.}
        \label{fig:difyieldnetwork_RR}
    \end{subfigure}
    \begin{subfigure}[b]{1\textwidth}
        \includegraphics[width = 1\textwidth]{figures/yield_dif_nodepropRed_River_Delta/yield_dif_nodepropRed_River_Delta.pdf}
        \caption[Three centrality measures of the nodes in differential co-occurrence network of rice injuries in different yield levels at Red River Delta, Vietnam]{Three centrality measures of the nodes in differential co-occurrence network of rice injuries in different yield levels at Red River Delta, Vietnam. A: node degree, B:clustering coefficient, and C:Betweenness.}
        \label{fig:nodepropdifyield_RR}
    \end{subfigure}
    \caption{Differential network analysis of survey data in different yield levels at Red River Delta, Vietnam}
    \label{fig:difyield_RR}
\end{figure}
 
 \begin{figure}
    \centering
    \begin{subfigure}[b]{1\textwidth}
        \includegraphics[width = 1\textwidth]{figures/difyieldTM/difyieldTM.pdf}
        \caption[Differential co-occurrence network of rice injuries in different yield levels at Tamil Nadu, India.]{Differential co-occurrence network of rice injuries in different yield levels at Tamil Nadu, India. The layout of the network graph is based on the Fruchterman-Reingold algorithm, which places nodes with stronger or more connections closer to each other.}
        \label{fig:difyieldnetwork_TM}
    \end{subfigure}
    \begin{subfigure}[b]{1\textwidth}
        \includegraphics[width = 1\textwidth]{figures/yield_dif_nodepropTamil_Nadu/yield_dif_nodepropTamil_Nadu.pdf}
        \caption{Three centrality measures of the nodes in co-occurrence network of rice injuries in dry season at Central Plain. A: node degree, B:clustering coefficient, and C:Betweenness.}
        \label{fig:nodepropdifyield_TM}
    \end{subfigure}
    \caption{Differential network analysis of survey data in different yield levels at Tamil Nadu, India.}
    \label{fig:yielddif_TM}
\end{figure}
 
  
 \begin{figure}
    \centering
    \begin{subfigure}[b]{1\textwidth}
        \includegraphics[width = 1\textwidth]{figures/difyieldWJ/difyieldWJ.pdf}
        \caption[Differential co-occurrence network of rice injuries in different yield levels at West Java, Indonesia.]{Differential co-occurrence network of rice injuries in different yield levels at West Java, Indonesia. The layout of the network graph is based on the Fruchterman-Reingold algorithm, which places nodes with stronger or more connections closer to each other.}
        \label{fig:difyieldnetwork_WJ}
    \end{subfigure}
    \begin{subfigure}[b]{1\textwidth}
        \includegraphics[width = 1\textwidth]{figures/yield_dif_nodepropWest_Java/yield_dif_nodepropWest_Java.pdf}
        \caption{Three centrality measures of the nodes in co-occurrence network of rice injuries in dry season at Central Plain. A: node degree, B:clustering coefficient, and C:Betweenness.}
        \label{fig:nodepropdifyield_WJ}
    \end{subfigure}
    \caption{Differential network analysis of survey data in different yield levels at West Java, Indonesia}
    \label{fig:yielddif_WJ}
\end{figure}


% ===
\begin{figure}[h]
    \centering
        \includegraphics[width = 1\textwidth]{figures/CP_yield_box.pdf}
        \caption{.}
        \label{fig:yield.box_CP}
\end{figure}


\begin{figure}
        \includegraphics[width = 1\textwidth]{figures/OD_yield_box.pdf}
        \caption{.}
\label{fig:yield.box_OD}
\end{figure}

\begin{figure}
    \centering
        \includegraphics[width = 1\textwidth]{figures/RR_yield_box.pdf}
        \caption{.}
        \label{fig:yield.box_RR}
\end{figure}


\begin{figure}
        \includegraphics[width = 1\textwidth]{figures/TM_yield_box.pdf}
        \caption{.}
\label{fig:yield.box_TM}
\end{figure}

\begin{figure}
        \includegraphics[width = 1\textwidth]{figures/WJ_yield_box.pdf}
        \caption{.}
\label{fig:yield.box_WJ}
\end{figure}


\clearpage
\subsection{DISCUSSION}

From the previous chapter, static dry season and wet season network were dominated by  associations of rice injuries that presented in farmers' fields based on survey data. The networks revealed that co-occurrence patterns were different across seasons. I applied differential network analysis based on quantifying the variability in the co-occurrence of rice injuries across seasons. Through network subtraction, however, the associations of injuries presenting in both seasons are removed, which allows sensitive detection of differentially represented conditions. Thus, network comparison reveal the whole patterns of co-occurrence particular tailored to the injury response to seasons.


%Further investigation showed that many differential interaction hubs were already well known to function as key components of DNA repair pathways (Fig. 2), which led us to predict that the remaining hubs might encode this role. Two such differential interaction hubs encode Slt2 and Bck1, mitogen-activated protein kinases (MAPKs) that have been implicated in the maintenance of cell wall integrity but not yet linked to DNA repair (Fig. 2). We found that Slt2 is both up-regulated and translocated to the nucleus upon MMS treatment, and it is required for appropriate regulation of ribonucleotide reductase genes in response to DNA damage (fig. S6, A to D) (14). Furthermore, both MAPKs show strong genetic interactions with DNA damage checkpoint genes (fig. S6, E and F), which suggests that they may function in a parallel signaling pathway.

%When sets of changed correlations have been identified, the next step is to establish the causal influences in the regulatory systems (i.e. to put directions on the edges in the undirected coexpression network) and, more importantly, to identify which causal influences have disappeared in the disease network with respect to the healthy network. Such disappeared regulatory mechanisms resulted in the observed changes in correlations and potentially could underlie the associated disease phenotype. Although it is not trivial to identify the causal system from the correlation patterns, the changes in correlation hint at the interesting regions of the network involved in disease which could form the basis for further detailed analysis. Systematic perturbations (e.g. experimental gene knockouts) are needed to establish the edges’ direction. Particular promise comes from so-called systems genetics experiments in which genotyping and gene expression data (and possibly metabolomics and proteomic

%To determine whether static untreated, static treated, or differential genetic networks best uncover DNA damage-response pathways, we examined a reference set of 31 known DNA repair genes (table S1). We noted that static networks were no more likely than random to include interactions with genes in this reference set (Fig. 1E). This lack of enrichment was observed in the untreated genetic network, as well as, surprisingly, the static network obtained under MMS. In contrast, the differential network—obtained through the quantitative difference in interaction across conditions—was highly enriched for interactions involving DNA damage-response genes such as RAD52, TEL1, and DUN1 (Fig. 1E) (10).


Further investigation showed that many differential interaction hubs were already well known to function as key components of DNA repair pathways (Fig. 2), which led us to predict that the remaining hubs might encode this role. Two such differential interaction hubs encode Slt2 and Bck1, mitogen-activated protein kinases (MAPKs) that have been implicated in the maintenance of cell wall integrity but not yet linked to DNA repair (Fig. 2). We found that Slt2 is both up-regulated and translocated to the nucleus upon MMS treatment, and it is required for appropriate regulation of ribonucleotide reductase genes in response to DNA damage (fig. S6, A to D) (14). Furthermore, both MAPKs show strong genetic interactions with DNA damage checkpoint genes (fig. S6, E and F), which suggests that they may function in a parallel signaling pathway.

%Systems biology methods like this are in high demand as the differences between many conditions, be they neurodegenerative diseases, brain diseases, different kinds of cancers, different degrees of disease severity, and so forth, are very subtle and cannot be easily highlighted using the usual off-the-shelf clustering or biological pathways identification algorithms. Many studies investigate only the genes that are unique to a condition, in order to analyze how different the conditions are. However, we hypothesize that even the genes that are common between conditions (conditions can be physiological, treatment, or time) can contribute to the differences between conditions either by invoking different biological pathways or by invoking the same biological pathways to varying degrees. Our differential network analysis method is applicable to other studies where a sequence of activities or processes is being determined. For instance, in our time-dependent analysis of low dose ionizing radiation study we were able to show the active biological processes at 3, 8, and 24 hours [12]. Our approach can aid in identifying the few genes that may be the key players in the specific condition and, therefore, potential biomarkers or therapeutic targets for that condition.

 