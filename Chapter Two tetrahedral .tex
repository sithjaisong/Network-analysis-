Farmers encountered an estimated average of 37 percent of their yield loss because of pests and disease. This proportion will be reduced when the farmers have efficient pest management. An important concept applied in pest management is ``disease triangle''\todo{Citation}, which explains that an occurrence of plant disease epidemic need three components, virulent pest, susceptible plant and conducive environment. At the farm level, we emphasize the interactions at population level. These interactions depend on not only the physical, biological environment but also man-made activities (Fig.\ref{fig:diseasetriangle}). According to \citet{Savary_2006_Quantification}, pest management tetrahedral first discussed by \cite{Zadoks_1979_Epidem} consists of four elements, a pest, crop, the environment and human. Human is recognized the important role in agroecosystem and should be considered to sustainable pest management \citep{Zadok_1985_Crop}. Humans are not limited to farmers, but also included to farmers' communities, social networks, agro-technology suppliers, food-chain stakeholders, research and extension, and policy-makers. 


Crop, C, incorporates elements pertaining to the host plant genetic make-up, including host plant resistance (HPR), the crop physiology, the crop phenology, and their interactions. C also incorporates microclimate factors that may influence the behaviour of crop-pest systems. This is because, while the microclimate in a crop is driven by physical meso-environment (under E), micro- climate also depends on the crop structure (i.e., its density and architecture). Thus, C not only accounts for the direct effects of HPR, but also for the conditions under which HPR may operate. Furthermore, the expression of resistance depends on the physi- ology of the crop (predisposition; Schoeneweiss, 1975). This is of prime importance for many plant pests in general, and in particular for rice diseases (such as brown spot, leaf blast, or sheath blight) and insect pests (such as planthoppers, leafhoppers, or leaf folders).
Summit H accounts for the indirect, yet often very strong, effect of crop management on pests (Palti, 1981; Zadoks, 1993). Summit H thus includes farmers and their management decisions, strategic or tactical (Zadoks and Schein, 1979). The former are made before crop establishment, including, e.g., a crop rotation, the choice of a vari- etydsusceptible or resistant to a pestd, or a crop establishment method. The latter are made after crop establishment, including, e.g., water and nutrient management, and of course pesticide use. This fourth element of the tetrahedron goes several scales beyond individual farmer’s field actions, however: it also encompasses policies, markets, subsidies, and decisions on which farmers have little or no control. Because agricultural policies are so strongly linked with perceptions and attitudes, whether individual or collective, H also involves psychologies, as well as possible con- flicting interests in different circles.
This framework allows consideration of the essential compo- nents that sustainable plant protection (IPM) must involve, and which interact through a set of key principles described in the following sections.

Each type of pests (animal pest, weeds, pathogens) are \todo{fix this section for clarity and grammar} often coexist and reduce yields in agricultural systems. They may have interactions. In addition to the individual effects on crops, these types of pests and cropping practices can interact and impact crop production. \cite{Ou_1985_Rice,Ho_1994_Weed,Cohen_1998_Importance} and \cite{Mew_2004_Looking} reviewed the relationships between crop management and pest incidence. 
 
The effective pest control management sound integrated and considerate as the system, IPM strategies are filed-dependent. Among the many reasons for this is the fact that the particular combination of pest and disease t (the injury profile) that affects given crop is, to a significant extent, driven by location-specific crop management practices. Another reason is that the amount of yield reduction (damage) that can be attributed to a given injury profile depend upon the attainable yield the crop would have achieved, had these pests been absent. 