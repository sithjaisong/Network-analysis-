Farmers encountered an estimated average of 37 percent of their yield loss because of pests and disease. This proportion will be reduced when the farmers have efficient pest management. An important concept applied in pest management is ``disease triangle''\todo{Citation}, which explains that an occurrence of plant disease epidemic need three components, virulent pest, susceptible plant and conducive environment. At the farm level, we emphasize the interactions at population level. These interactions depend on not only the physical, biological environment but also man-made activities (Fig.\ref{fig:diseasetriangle}). According to \citet{Savary_2006_Quantification}, pest management tetrahedral first discussed by \cite{Zadoks_1979_Epidem} consists of four elements, a pest, crop, the environment and human. Human is recognized the important role in agroecosystem and should be considered to sustainable pest management \citep{Zadok_1985_Crop}. Humans are not limited to farmers, but also included to farmers' communities, social networks, agro-technology suppliers, food-chain stakeholders, research and extension, and policy-makers. 



Crop, C, incorporates elements pertaining to the host plant genetic make-up, including host plant resistance (HPR), the crop physiology, the crop phenology, and their interactions. C also incorporates microclimate factors that may influence the behaviour of crop-pest systems. This is because, while the microclimate in a crop is driven by physical meso-environment (under E), micro- climate also depends on the crop structure (i.e., its density and architecture). Thus, C not only accounts for the direct effects of HPR, but also for the conditions under which HPR may operate. Furthermore, the expression of resistance depends on the physi- ology of the crop (predisposition; Schoeneweiss, 1975). 

The interactions between human activities and pathogens are considered indirect, but often very strong. For example, crop establishment (e.g., a crop rotation, the choice of a varieties, a crop establishment method), water and nutrient management, and pesticide use are strongly linked to the pest and disease outbreak \cite{Zadoks_1979_Epidem}. This is of importance to design effective pest control management, which is involved the sustainable plant protection (IPM). Moreover,IPM strategies are field-dependent due to the fact that the combination of pests and diseases is vary from different location and crop management practices \cite{Mew_2004_Looking, Savary_2012_Review}. According to \cite{Savary_2000_Quantification, Savary_2000_Characterization}, The amount of yield reduction varies from the different the injury profiles.