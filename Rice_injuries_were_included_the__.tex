Rice injuries were included the injuries caused by animal pests, pathogens, and also weeds, which are harmful to rice plants, and importantly considered to reduce yield productivity. The data collected at the field level from individual farmers. They were evaluated over time. Each surveyed field was regarded to be the representatives with a characteristic set of attributes, which were related to cropping practices, and environments. Although a very large number of pathogens, insects and weeds are harmful to rice, many are seldom considered to cause yield losses (Teng, 1990b; Willocquet et al., 2004). Data were derived from time-dependent information on diseases, insects and weeds observed at four development stages of the growing crop: tillering, booting, early dough and maturity in such a way that they would best account for possible yield reductions. Data were grouped in the field assessment procedure according to their mechanisms. Injuries on leaves such as bacterial leaf blight (BLB), bacterial leaf streak (BLS), leaf blast (LB), brown spot (BS) were determined as a proportion of injured leaves. Injuries on tillers or hills such as stem rot, sheath rot, sheath blight, whitehead, deadheat were determined as a proportion of injured tillers or panicles. Weed infestation was determined as the percent of area are covered by any weed species above the rice canopy (WA) and the weed below the rice canopy (WB). Before analysis, data were compacted over time during crop growth. Two types of data were computed, depending on the nature of injury \citpe{Savarysurvey2009}. One is an area under injury progress curve (AUIPC) used for injury variables, which present on the leaves. Another is the maximum level at any of the four observations used for injury variables that can be observed on tillers, panicles, and hills. The area under injury progress curve (AUIPC) \citep{Campbell_1990_Introduction} were calculated by the mid- point method using the following equation: The area under injury progress curve (AUIPC) was calculated by the mid- point method using the following equation::

\begin{equation}
AUIPC = \sum{\frac{1}{2(X_{i} + X_{i-1})(T_{i} - T_{i-1})}}
\end{equation}

where $X_i$ is percentage (\%) of leaves, tillers or panicles injured due to rice pests (e.g., leaf blast, leaf folder), or number of insects (e.g., plant hoppers, leaf hoppers) per quadrat, or percentage (\%) of weed infestation (ground coverage) at the $i$th observation, $T_i$ is time in rice development stage units (dsu) on a 0 to 100 scale (10: seedling, 20: tillering, 30: stem elongation, 40: booting, 50: heading, 60: flowering, 70: milk, 80: dough, 90: ripening, 100: fully mature) at the $i$th observation and $n$ is total number of observations.

