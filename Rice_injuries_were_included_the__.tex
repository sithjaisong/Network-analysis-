Rice injuries were included the injuries caused by animal pests, diseases, and also weed infestation, which are harmful to rice plants, and importantly considered to reduce yield productivity. They were grouped in the field assessment procedure according to their mechanisms. Injuries on leaves such as bacterial leaf blight (BLB), bacterial leaf streak (BLS), leaf blast (LB), brown spot (BS) were determined as a proportion of injured leaves. Injuries on tillers or hills such as stem rot, sheath rot, sheath blight, whitehead, dead-heat were determined as a proportion of injured tillers or panicles. Weed infestation was determined as the percent of weed above the rice canopy (WA) and the weed below the rice canopy (WB). Even though the many injuries happened in the rice fields, some diseases such as leaf scald, leaf smut, and insects such as leaf miner, rice thrips, rice hispa, and defoliators in general are not considered as major yield-reducing factors.

The area under injury progress curve (AUIPC) was calculated by the mid- point method using the following equation:

\begin{equation}
AUIPC = \sum{\frac{1}{2(X_{i} + X_{i-1})(T_{i} - T_{i-1})}}
\end{equation}

where $X_i$ is percentage (\%) of leaves, tillers or panicles injured due to rice pests (e.g., leaf blast, leaf folder), or number of insects (e.g., plant hoppers, leaf hoppers) per quadrat, or percentage (\%) of weed infestation (ground coverage) at the ith observation, Ti is time in rice development stage units (dsu) on a 0 to 100 scale (10: seedling, 20: tillering, 30: stem elongation, 40: booting, 50: heading, 60: flowering, 70: milk, 80: dough, 90: ripening, 100: fully mature) at the ith observation and n is total number of observations.