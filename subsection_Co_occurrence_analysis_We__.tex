\subsection*{Co- occurrence analysis}

We considered co-occurrence both of positive and negative correlations based on Spearman's rank based correlation between pairs of pest injuries within each dataset with the strength of relationship ($\rho$ from Spearman's correlation) represented by the correlation coefficient. The coefficients with $p$-values less than $p$ = 0.05 were considered. Negative correlations (indicative of anti-co-occurrence relationship) were also included in analysis. The \texttt{R} function \texttt{cor.test} with parameter method `Spearman' (package stats) was used for calculate Spearman's correlation coefficient ($\rho$), which is defined as the Pearson correlation coefficient between the ranked variables.
These correlation relationships were generated for each pair of injury within each location replicate as long as both injury had incidence value greater than 0. We made a network of co-occurrence relationships within each locations based on the strength of the correlation ($\rho$ from the Spearman's correlation), and co-occurrence relationships were only included if they occurred across all locations. Though this method has been illustrated to produce some spurious co-occurrence relationships among data, this rank-based correlation statistic does not require any transformation of variables to fit assumptions of normality and may outperform Pearson’s correlations. To increase our level of stringency that may reduce the appearance of spurious co-occurrences within our networks, pairwise relationships had to be consistent across all datasets of a given ecosystem type, greatly reducing the number of co-occurrence pairs.
