\subsection{Materials and Methods}
I used survey data were collected from farmers' fields ,analyzed co-occurrence relationships of rice injuries, and performed network analysis as described previously. Actual yield estimates were collected from each farmers' fields surveyed. Before analyzing. Three yield levels were  I grouped the survey data into 6 groups by different yield levels and seasons.  

\textbf{Network Comparison Measures}
For the pair of $x_{i}$ injury and $y_{i}$ variable, I denoted the correlation coefficient based on Spearman's correlation coefficient by $c_{xy}^1$ and $c_{xy}^2$ in networks 1 and 2, respectively. To test whether the 2 correlation coefficients were significantly different of each network, I used Spearman's correlation coefficients, the p-values of the correlation test, the difference of the 2 correlations, the corresponding $p$-values, and the result of Fisher's z-test. First, I applied the Fisher $z$-transformation in order to stabilize variances due to sample size.

\begin{equation}
z_{xy} = \frac{1}{2} \log\left[{\frac{1 + c_{xy}}{1 - c_{xy}}}\right]
\end{equation}

if we let $z_{xy}^D $ and $z_{xy}^W$ denote the $z$- transformation for dry and wet season variable pairs, respectively. 
Next, differences between the two correlations can be test using following. 

\begin{equation}
\triangle z_{xy} = \frac{z_{xy}^D - z_{xy}^W}{\sqrt{\frac{1}{N_{D}-3}+ \frac{1}{N_{W}-3}}}
\end{equation}

$N_{D}$ and $N_{W}$ represent the sample size for each of season network for each country. The $Z$ has an approximately Gaussian distribution under null hypotheses that the population correlations are equal. The pair-wise correlation significants are considered at $p$vlaue < 0.05.


 

