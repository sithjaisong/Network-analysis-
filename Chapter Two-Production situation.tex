\subsection*{Crop establishment and pests}

Crop establishment methods of rice are various, but mainly can be categorized into two types, direct seeding and transplanting. 

Weed composition in the rice field is constantly changing. Some weed species are favorably affected by production practice while some are adversely affected. Direct seeded rice more possibly encounters with weed competition than transplanted rice because they emerge simultaneously with rice seedlings and because of the absence of flooding in the early stages. Generally, weeds such as grasses, sedges, and broadleaf weeds are found in direct seeded rice fields, which dominant weeds are \textit{Echinochloa crus-galli}, and \textit{Leptochloa chinensis} among grasses, \textit{Cyperus difformis}, and \textit{Fimbristylis miliacea} among sedges, and \textit{Ammania baccifera}, \textit{Eclipta prostrata}, and \textit{Sphenoclea zeylanica} in the broadleaf category \cite{Juraimi_2013_Sustainable}.

Compared with transplanted rice, the occurrence of insect pests and diseases is more intense in direct seeded rice because of high plant density and the consequent cooler, more humid, and shadier microclimate inside the canopy \cite{Pandey_2002_Direct}, are conducive to different epidemiological conditions \cite{Willocquet_2000_Effect}. The major insect pests of direct seeded rice are brown planthoppers, stem borers, leaffolders, and gall midges. Important diseases that affect direct seeded rice are blast, ragged stunt disease, yellow orange leaf disease, sheath blight, and dirty panicle \citep{Pongprasert_1995_Insect}. In addition to insect pests and diseases, other pests that attack emerging rice seedlings are the golden apple snail (\textit{Pomacea canaliculata}) and rats, which are more serious problems in direct seeded rice than transplanted rice \cite{Pandey_2002_Direct}.

\subsection*{Nutrient management and pest occurrence}

Nutrition management is one of the most important practices of a high production system, but nutrition management will affect the response of rice to pests, as well as development pattern of pest populations due to the change of conditions \cite{Zadoks_1979_Epidem, Willocquet_2000_Effect}. Soil fertility practices can impact the physiological susceptibility of crop plants to insect pests by either affecting the resistance of individual plant to attack or by weakening plant vulnerability to certain pests. Some studies have also documented how the shift from organic soil management to chemical fertilizers has increased the potential of certain insects and diseases to cause economic losses \cite{Castilla_2003_Interaction}.

Nitrogen, phosphorus and potassium are most often managed by the addition of fertilizers to soils. The others are most often found in sufficient quantities in most soils and no soil amendments are needed to ensure adequate supply unless soil pH limits them. However, I will briefly summarize the effects of nitrogen, phosphorus, and potassium with their relationship to some of the most important rice diseases and insects.

Nitrogen can prolong the vegetative period and increases the proportion of young to mature tissues in rice plants. It can also reduce the amount of cellulose in plant cell walls, predisposing plants to lodging. Aside from these, increased nitrogen was found to reduce phytoalexins in plant cells. Phytoalexins are anti-microbial compounds that build up in plants as a result of infection or stress, and are associated with resistance to fungal and bacterial diseases. On the other hand, if applied correctly, nitrogen allows better plant compensation and tolerance to injury. Bacterial leaf blight can be aggravated by too much nitrogen, and it is even more problematic during the wet season. This is one reason why the recommendation for nitrogen fertilizer during the wet season is lower compared to the recommendation during the dry season. Other than bacterial leaf blight, excessive nitrogen can also be conducive to the development of many other diseases, such as bakanae, bacterial leaf streak, false smut, leaf blast, sheath blight, and sheath rot \cite{Castilla_2003_Interaction}.

As to the effect of nitrogen on the disease, excessive nitrogen application enhanced occurrence of insect pest, especially stem borers, whorl maggots, brown planthoppers and leaffolders \citep{Chau_2003_Impacts, Litsinger_2011_Cultural, Rashid_2014_Effect}. Except whorl maggot and rice thrips, their population did not increase following as the use of nitrogen increase \citep{Chau_2003_Impacts}.

\subsubsection{The effects of phosphorous on pest occurrence}

Just as important as nitrogen, phosphorous is another essential nutrient element that promotes vigorous root development and is responsible for hardening plant tissue. Phosphorus also shortens vegetative period of the plant, as opposed to the effect of nitrogen. Proper timing in the application of phosphorus may lower the incidence of brown spot and leaf blast \cite{Castilla_2003_Interaction}.

In relation to disease incidence, what we know is that phosphorus can reduce the occurrence of bacterial leaf blight and brown spot, but may promote leaf blast and sheath blight of rice. The effect of phosphorus on the population of rice insect pests was not strong \citep{Chau_2003_Impacts, Rashid_2014_Effect}, but the results of \citet{Rashid_2014_Effect} showed that increased phosphorus applicant enhanced the population of brown planthoppers.

\subsubsection{The effects of potassium on pest occurrence}

Among the three essential nutrients mentioned, potassium appeared to be the most consistent and effective in minimizing disease incidence. It was found to lower the infection rate of bacterial leaf blight, leaf blast, and brown spot, sheath blight, sheath rot, stem rot and narrow brown spot \cite{Ou_1985_Rice}. This is because potessium is found to increase the concentration of inhibitory amino acids in the plants as well as phytoalexins \cite{Castilla_2003_Interaction}. Potassium also hardens the plants tissues which can minimize lodging incidence and support faster recovery of injured or stressed plants \cite{Harrewijn_1979}. There were not many studies about the effect of potassium on the rice insect pests. \citet{Rashid_2014_Effect} showed the result in Bangladesh that high potassium application decreased population build up and dry weight of brown plant hoppers and \cite{Salim_2002_Effects} reported that reported that deficiency of potassium in rice plants increased population build up of white backed planthopper and application of high dose of potassium to rice plants decreased population build up of the insect.