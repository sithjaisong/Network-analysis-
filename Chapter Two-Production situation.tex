\subsection*{Crop establishment and pests}
\addcontentsline{toc}{subsection}{Crop establishment and pests}

Crop establishment methods of rice are various, but mainly can be categorized into two types, direct seeding and transplanting. 

Weed composition in the rice field is constantly changing. Some weed species are favorably affected by production practice while some are adversely affected. Direct seeded rice more possibly encounters with weed competition than transplanted rice because they emerge simultaneously with rice seedlings and because of the absence of flooding in the early stages. Generally, weeds such as grasses, sedges, and broadleaf weeds are found in direct seeded rice fields, which dominant weeds are \textit{Echinochloa crus-galli}, and \textit{Leptochloa chinensis} among grasses, \textit{Cyperus difformis}, and \textit{Fimbristylis miliacea} among sedges, and \textit{Ammania baccifera}, \textit{Eclipta prostrata}, and \textit{Sphenoclea zeylanica} in the broadleaf category. \todo{Citation}

Compared with transplanted rice, the occurrence of insect pests and diseases is more intense in direct seeded rice because of high plant density and the consequent cooler, more humid, and shadier microclimate inside the canopy\todo{Citation}. The major insect pests of direct seeded rice are brown planthoppers, stem borers, leaffolders, and gall midges. Important diseases that affect direct seeded rice are blast, ragged stunt disease, yellow orange leaf disease, sheath blight, and dirty panicle \citep{Pongprasert_1995_Insect}. In addition to insect pests and diseases, other pests that attack emerging rice seedlings are the golden apple snail (\textit{Pomacea canaliculata}) and rats, which are more serious problems in direct seeded rice than transplanted rice.\todo{Citation}


\subsection*{Nutrient management and pest occurrence}
\addcontentsline{toc}{subsection}{Nutrient management and pests}

Nutrition management is one of the most important practices of a high production system, but nutrition management will affect the response of rice to pests, as well as development pattern of pest populations due to the change of environments\todo{Change of what environments?}. Soil fertility practices can impact the physiological susceptibility of crop plants to insect pests by either affecting the resistance of individual plant to attack or by weakening plant vulnerability to certain pests. Some studies have also documented how the shift from organic soil management to chemical fertilizers has increased the potential of certain insects and diseases to cause economic losses.\todo{Multiple citations}

Nitrogen, phosphorus and potassium are most often managed by the addition of fertilisers to soils. The others are most often found in sufficient quantities in most soils and no soil amendments are needed to ensure adequate supply unless soil pH limits them. However, I will briefly summarize the effects of nitrogen, phosphorus, and potassium with their relationship to some of the most important rice diseases and insects.

\subsubsection{The effects of nitrogen on pest occurrence}\todo{Occurence or injury or yield losses?}

Nitrogen can prolong the vegetative period and increases the proportion of young to mature tissues in rice plants. It can also reduce the amount of cellulose in plant cell walls, predisposing plants to lodging. Aside from these, increased nitrogen was found to reduce phytoalexins in plant cells. Phytoalexins are anti-microbial compounds that build up in plants as a result of infection or stress, and are associated with resistance to fungal and bacterial diseases. On the other hand, if applied correctly, nitrogen allows better plant compensation and tolerance to injury.\todo{Citation}

Bacterial leaf blight can be aggravated by too much nitrogen, and it is even more problematic during the wet season. This is one reason why the recommendation for nitrogen fertilizer during the wet season is lower compared to the recommendation during the dry season. Other than bacterial leaf blight, excessive nitrogen can also be conducive to the development of many other diseases, such as bakanae, bacterial leaf streak, false smut, leaf blast, sheath blight, and sheath rot.\todo{Citation}

As with the impact of nitrogen on the disease, excessive nitrogen application was force more occurrence of insect pest especially stem borers, whorl maggots, brown planthoppers and leaffolders \citep{chau2003impacts,litsinger2011cultural,rashid2014effect}\todo{Revise for clarity}. The effect of nitrogen on rice insect pests are ambiguous. For example, the number of whorl maggot and rice thrips population were not develop following as the dose of nitrogen increase \citep{chau2003impacts}\todo{Revise for clarity}.

\subsubsection{The effects of phosphorous on pest occurrence}

Just as important as nitrogen, phosphorous is another essential nutrient element that promotes vigorous root development and is responsible for hardening plant tissue. Phosphorus also shortens vegetative period of the plant, as opposed to the effect of nitrogen. Proper timing in the application of phosphorus may lower the incidence of brown spot and leaf blast.\todo{Citation}

In relation to disease incidence, what we know is that phosphorus can reduce the occurrence of bacterial leaf blight and brown spot, but may promote leaf blast and sheath blight of rice. The effect of phosphorus on the population of rice insect pests was not strong \citep{Chau_2003_Impacts,Rashid_2014_Effect}, but the results of \cite{Rashid_2014_Effect} showed that increased phosphorus applicant enhanced the population of brown planthoppers.

\subsubsection{The effects of potassium on pest occurrence}

Among the three essential nutrients mentioned, potassium appeared to be the most consistent and effective in minimizing disease incidence. It was found to lower the infection rate of bacterial leaf blight, leaf blast, and brown spot, sheath blight, sheath rot, stem rot and narrow brown spot\todo{Citation}. This is because K is found to increase the concentration of inhibitory amino acids in the plants as well as phytoalexins\todo{Citation}. Potassium also hardens the plants tissues which can minimize lodging incidence and support faster recovery of injured or stressed plants\todo{Citation}. There were not many studies about the effect of potassium on the rice insect pests. \cite{Rashid_2014_Effect} showed the result in Bangladesh that high potassium application decreased population build up and dry weight of brown plant hoppers and \cite{salim2002effects} reported that reported that deficiency of potassium in rice plants increased population build up of white backed planthopper and application of high dose of potassium to rice plants decreased population build up of the insect.

\subsection*{Interactions between coexisting rice pests}

Mixed populations of brown planthoppers and whitebacked planthoppers were reported in many areas. \cite{naganagoud2010studies} reported the experiment multiple pest yield loss to damage function, and economic injury level Iso-loss equation by mutual regression of incidence of two pests. Leaffolder negatively related to stem borers. \cite{selvaraj2012determination} report the population of four major rice insect pests (brown planthoppers, whitebacked planthoppers, stem borers, and leaffolders) by light traps in Tungabhadra area, India from 1988 to 1990. The observations revealed that the mean populations of these pests decreased in 1989 and increased in 1990\todo{Revise para for clarity}.

Many plant diseases are caused by a particular species or even strain in diverse environments. However, there are a number of disease complexes where no one organism causes the full set of symptoms.

Akioshi disease is known as ``hydrogen sulfide toxicity'', causes black crown and root rot, which affect the nutrient absorption. The disease is associated with abnormal soils (silica, potassium, magnesium deficiency). Plants affected by Akiochi become susceptible to \textit{Bipolaris oryzae}, the pathogen of brown spot, which is used as an index of Akiochi \citep{ouricedisease}.

The dirty panicle (grain discoloration) are complex disease caused my many fungal pathogens on the glumes, kernels or both. The fungi that are reported to be associated with discoloration of grains are \textit{Bipolaris oryzae}, \textit{Alternaria padwickii}, \textit{Pyricularia oryzae}, \textit{Fusarium moniliforme}, \textit{F. graminearum}, \textit{Nigrospora oryzae} and \textit{Curvularia} spp.\cite{ouricedisease}. However, it is still conflict such as the Pyriculris test the seed from a filed heavily infested by blast, but no \textit{P. oryzae} isolated\todo{Revise for grammar and clarity}.

The relationships between insect and disease are in the term of vectors\todo{Unclear}. Insects are the vectors of important rice diseases, mostly are viral disease\todo{Revise for grammar}. The incidence of tungro depend on vector population and composition (different species of vector differ in transmission ability, availability of virus source, since the virus is nonpersistence, susceptibility and age of host plant, synchronization of these three factor\todo{Revise for grammar}. Weed hosts of the vectors and of the virus play a role and climates affect all these factors \cite{naganagoud2010studies}. Some found the tungro with yellow dwarf disease \cite{ouricedisease}\todo{Revise for grammar}

\subsection*{Survey in the field of epidemiological research}\todo{Revise for grammar}

The rice crop system is complex that the number of components is large and their interaction are complex\todo{Revise for grammar}. The dynamic of the system is encompassed by a growing crop, its physical environment, its diseases and pests, and farmers' actions, which influence the whole system. The number of rice pests (diseases, insects, and weeds) to be considered and the levels at which they vary from field to field. A rationale is often deeded to delineate the limits of the system to be addresses\todo{Revise for grammar}.

A survey is a useful tool to provide an adequate account of this diversity and deal with the several pests or variables affecting crop production, which are essential components of systems research in plant protection \citet{Zadoks_1979_Epidem}. Field surveys generate information at the level of the population of crop stands rather than of the individual plot, field, or orchard. Such surveys raise questions regarding variation in injury and its relationships to crop management, crop environment, crop performance, and the occurrence of multiple injuries.


Application of surveys can be found in a huge diversity of research fields, including crop characteristic, cropping practices and single disease or multiple disease and insect. The pest are measurable biological variable that can be used for characterize the injuries profiled (disease and insect injuries syndrome) to stratify farmer's fields account into the characteristic of their production situations.

\cite{Savary_1995_Use} emphasized field surveys as the basic means for acquiring information and expanding knowledge to higher scales. Conversely, field surveys and the associated analytical approaches provide a framework in which to characterize disease intensities and their relationships with attributes (including crop management) of the environments where injuries occur; but they usually do not generate crop loss data \textit{per se}.


The concept of production situation described the set of factors - physical, biological, and socioeconomic - that determine agricultural production\todo{Citation}. This concept is used here within the restricted scope of an individual rice field and from the limited view point of plant protection. A quick overview of the variables listed in the following survey procedure indicates how this concept is operationalized in a survey: this list includes a few key components of rice crop management (which reflect the physical and socioeconomic environments of a crop and their interactions) and a series of rice pests (that is, insects, pathogens, and weeds).

This approach has the considerable advantage of providing quantitative estimate of the contribution of each constraint to the total yield loss. They study may be considered among the first series, where a set of pest constraints of a crop, simultaneously with environmental variables, and variables representing the current cropping practices were considered. Pests only considered some of the ``yield determining variables''. the approaches may also be considerate an attempt to incorporated the concept of production situation \textit{i.e.} the set of environment that lead to a given yield output of the system\todo{Revise for grammar}. 


\subsection*{Common techniques for analysis of survey data}

Methodologies used to interpret the survey data are developed for studies of the relationships for injuries - yield losses. The development of theses studied started focusing on single-disease (pest) pathosystem to more complex. The classic analytical approaches aim to assess group-wise differences, either in a univariate \textit{i.e.} parameter-by-parameter fashion or using multivariate techniques (\textit{e.g.} multivariate analysis of variance (MANOVA), correspondence analysis (CA), multiple correspondence analysis (MCA), or principle component analysis (PCA)). Univariate methodologies are frequently used to reduce a possibly large number of measured analyses to only those that show the strongest response under the investigated conditions. Examples for such univariate approaches are significance testing methods and simple linear regression model. However, univariate methods fail to apply to different production situation or the scope that model can be applied.


Therefore, multivariate analysis methods seek to capture not only changes of a single pest. Probably the most prominent multivariate analysis techniques applied in the field of survey data are principal component analysis, cluster analysis and correspondence analysis. These analyses do not always; and to yield loss estimate\todo{Revise for clarity}. They can inform us the close relationship between changing production situations and shift crop health syndromes (changes in the dynamics of group disease or crop-limiting factors in an area)\todo{Revise for clarity}. 

Farmers' fields survey datasets are of immense value to plant pathologists; they, however, are generally heterogeneous in format (being qualitative or quantitative) and/or precision. A number of methods, some now old and seldom used, and other more recent and comparatively easier to use, are available to process this type of information. The approached used to analyst these data included three main steps: cluster analysis, multiple correspondence analysis and logistic regression.


Cluster analysis represents another unsupervised multivariate technique suitable for the analysis of survey data with hierarchical cluster analysis and $k$-means clustering being the most prominent representatives. In general, clustering methods group and visualize samples according to intrinsic similarities in their measurements, irrespective of sample groupings. 
