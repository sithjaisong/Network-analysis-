\subsection*{Common techniques for analysis of survey data}

Methodologies used to interpret the survey data are developed for studies of the relationships for injuries - yield losses. The development of theses studied started focusing on single-disease (pest) pathosystem to more complex. The classic analytical approaches aim to assess group-wise differences, either in a univariate \textit{i.e.} parameter-by-parameter fashion or using multivariate techniques (\textit{e.g.} multivariate analysis of variance (MANOVA), correspondence analysis (CA), multiple correspondence analysis (MCA), or principle component analysis (PCA)). Univariate methodologies are frequently used to reduce a possibly large number of measured analyses to only those that show the strongest response under the investigated conditions. Examples for such univariate approaches are significance testing methods and simple linear regression model. However, univariate methods fail to apply to different production situation or the scope that model can be applied.


Therefore, multivariate analysis methods seek to capture not only changes of a single pest. Probably the most prominent multivariate analysis techniques applied in the field of survey data are principal component analysis, cluster analysis and correspondence analysis. These analyses do not always; and to yield loss estimate\todo{Revise for clarity}. They can inform us the close relationship between changing production situations and shift crop health syndromes (changes in the dynamics of group disease or crop-limiting factors in an area)\todo{Revise for clarity}. 

Farmers' fields survey datasets are of immense value to plant pathologists; they, however, are generally heterogeneous in format (being qualitative or quantitative) and/or precision. A number of methods, some now old and seldom used, and other more recent and comparatively easier to use, are available to process this type of information. The approached used to analyst these data included three main steps: cluster analysis, multiple correspondence analysis and logistic regression.


Cluster analysis represents another unsupervised multivariate technique suitable for the analysis of survey data with hierarchical cluster analysis and $k$-means clustering being the most prominent representatives. In general, clustering methods group and visualize samples according to intrinsic similarities in their measurements, irrespective of sample groupings. 