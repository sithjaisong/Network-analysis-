\subsection{Introduction}

Agricultural crop plants are frequently injured, and infected by more than one species of pests and pathogens at the same time. Many of these injuries may affect yields. Because of this co-occurrence in injuries, the idea of ``crop health" has been highlighted and implemented to manage this combination of injuries or so-called injury profiles \citep{Savary_2006}. Co-occurrence patterns of injuries are beginning to provide important insight into these injury profiles, which possibly present co-occurring or anti-co-occurring (mutually exclusive) relationships between injury-injury. Uncovering these patterns is important to implications in plant disease epidemiology and management. However, there are only a few reports of injury–injury relationships in rice crop systems are currently unknown. This could be a difficult task since complex patterns of injury profiles are related to environmental conditions, cultural practices, and geography \citep{Willocquet_2008_Simulating}.

To address this issue, we used in-field surveys as a tool to develop ground-truth databases that allowed us to identify the major yield reducing pests in irrigated lowland rice ecosystems. These sorts of databases provide an overview of the complex relationships between the crop, its management, pest injuries, and yields. Several previous studies \cite{Savary_2000_Characterization,Savary_2000_Quantification,Dong_2010_Characterization,Reddy_2011_Characterizing} involved surveys that were used to characterize injury profiles in an individual production situation (a set of factors including cultural practices, weather condition, socioeconomics, \textit{etc}.) that determine agricultural production, and the injury profiles using nonparametric multivariate analysis such as cluster analysis, correspondence analysis, or multiple correspondence analysis. Their results led to the conclusions that injury profiles (the combination of disease and pest injury that may occur in a given farmer’s field) were found that the patterns of injury profiles were found co-occurrence patterns across sites, which are associated at regional scale. For example, stem rot, sheath blight, planthopper, and rice whorl maggot injuries, are high incidence, with low incidence of brown spot, and absence of bacterial leaf blight, leaf blast, and neck blast are a common pattern in tropical Asia from the study of\citet{Savary_2000_Characterization}.

Co-occurrence analysis and network theory have recently been used to reveal the patterns of co-occurrence between microorganisms in the complex environments ranging from human gut to ocean and soils \citep{Faust_2012_Microbial_co,Ma_2016_Geographic}. Co-occurrence patterns are ubiquitous and particularly important in understanding community structure, offering new insights into potential interaction network. Recent reviews of network based approaches reveal that these tools have demonstrated previously unseen co-occurrence patterns, such as strong non-random association, Topology based analysis of large networks has been proven powerful for studying the characteristics of co-occurrence pattern of the communities, and identify keystone species in ecological community \citep{Williams_2014_demonstrating,Barberan_2012_Network}, or key actors in social networks \citep{Crowston_2006_Hierarchy}. Here, we significantly advance this study by providing a comprehensive understanding of the topological shifts of animal pest injury and disease co-occurrence networks at regional scale.

South and Southeast Asia represent big bowl of rice for the world population. Comparing the topological properties of the node associated with farmer occurrence in the different countries and examining network level topological features can provide us with insight into variation in the co-occurrence patterns in different countries. this approach helps contextualize the animal pests -- disease association by taking to account the complex network of potential association among animal pest and disease occurring in farmers's fields in theses countries.Specifically, we addressed the following questions:(i) Do the topological features of co-occurrence network vary between different countries.(ii) Do the animal pests or diseases exhibit different co-occurrence patters from different countries. (iii) What animal pests or diseases play the important role in co-occurrence network in order to target to control or monitor. To answer these questions, we performed crop health survey at the farmers's fields in different five countries in South and Southeast Asia and implemented co-occurrence network analysis to examine the topological feature differences  across countries. Our main objective was to characterize and better understand co-occurrence network patterns in the association of rice animal pest and diseases.