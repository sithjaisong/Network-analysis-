\subsection{Introduction}

Agricultural crop plants are frequently injured, and infected by more than one species of pests and pathogens at the same time. Many of these injuries may affect yields. Because of this co-occurrence in injuries, the idea of ``crop health" has been highlighted and implemented to manage this combination of injuries or so-called injury profiles \citep{Savary_2006}. Co-occurrence patterns of injuries are beginning to provide important insight into these injury profiles, which possibly present co-occurring or anti-co-occurring (mutually exclusive) relationships between injury-injury. Uncovering these patterns is important to implications in plant disease epidemiology and management. However, there are only a few reports of injury–injury relationships in rice crop systems are currently unknown. This could be a difficult task since complex patterns of injury profiles are related to environmental conditions, cultural practices, and geography \citep{Willocquet_2008_Simulating}.

To address this issue, we used in-field surveys as a tool to develop ground-truth databases that allowed us to identify the major yield reducing pests in irrigated lowland rice ecosystems. These sorts of databases provide an overview of the complex relationships between the crop, its management, pest injuries, and yields. Several previous studies \cite{Savary_2000_Characterization,Savary_2000_Quantification,Dong_2010_Characterization,Reddy_2011_Characterizing} involved surveys that were used to characterize injury profiles in an individual production situation (a set of factors including cultural practices, weather condition, socioeconomics, \textit{etc}.) that determine agricultural production, and the injury profiles using nonparametric multivariate analysis such as cluster analysis, correspondence analysis, or multiple correspondence analysis. Their results led to the conclusions that observed injury profiles were strongly associated with production situations and the level of actual yields. For example, hight incidence of stem rot and sheath blight are frequently found together with high mineral fertilizer inputs, low pesticide use, and good water management in transplanted rice crops.

We used the technique from an ecological study to the co-occurrence analysis and network theory to reveal the patterns of co-occurrence of injury profiles.  While existing statistic approaches for nonparametric multivariate analysis the survey data have been used, the methods presented in this paper have the advantages of understanding of the entire on-occurrence patterns of past injuries, identification of potential connections between many injuries.

Network theory is the study of relationships between entities (nodes) and connections between these entities (edges). Network theory has been widely used to describe social and biological datasets \cite{Moslonka_Lefebvre_2011}, and it has been shown to be a useful tool for systemically modeling and  interpretation as a graph or network. Here, we consider pest injuries as nodes and create an edge between any two injuries if they are co-existing. We give an edge greater weight if the two injuries have strong co-occurrence at either end. 

The aim of this study was to analyze the structure of co-occurrence network of pest injuries at different countries by quantifying the important aspects of the position of the specific pest injury, information based on network theory, which help to understand the formation and characteristics of co-occurrence patterns. Furthermore, key factors analysis could infer the important injuries to be controlled or monitored.