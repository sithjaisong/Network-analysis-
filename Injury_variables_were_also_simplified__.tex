Injury variables were also simplified. Although a very large number of pathogens, insects, and weeds are harmful to rice, many are seldom considered to cause yield losses. Diseases such as narrow brown spot, bacterial leaf streak, leaf scald, and leaf smut, and insects such as rice bugs, rice hispa, and defoliators in general are not considered to represent major, widespread, yield-reducing factors. The study therefore concentrated on injuries listed in . A second aspect pertains to the injury mechanisms, and Table 1 includes injuries that were grouped in the field assessment procedure according to their nature: photosynthetic area stealers (BLB, BS, LB: proportion of injured leaves), senescence accelerators (BLB, SHB, LB: proportion of injured leaves, except for SHB), tissue users (leaves: RWM, LF: proportion of injured leaves; tillers: SR, SHB, DH: proportion of injured tillers; panicles: SHR, WH: proportion of injured panicles), assimilate sappers (PH: number of insects sampled), turgor reducers (at the tiller level: SR, SHB: proportion of injured tillers; at the panicle level: NB: proportion of injured panicles), and stand reducers (WA and WB).