\subsection{Discussion}

Rice injuries were found commonly in South and South east Asia, but at different levels of incidence. 
Some injuries of this study is relatively low prevalence of areas that have been reported such as leaf blast, brown spot. It could be implied that these injuries strongly depended on locations or climatic conduction to develop, so they were not observed at all locations or seasons during survey were conducted. Another reason is that there are some factors such as the utility of resistance verities in the farmer’s fields surveyed, so we could observe those injuries at low level of incidence. The similar reasoning could explain to many injuries such as BLB, RB, NB, which widely occur in rice growing areas.

From the survey data of rice injuries observed in farmers’ fields, I analyzed the interaction and build the network based on that data. The methods applied for building a co-occurrence network of rice injuries were adapted from ecological studies. Usually, relationships were assessed using Pearson correlation. However, the use of the Pearson correlation coefficient is problematic because it requires the variables are applied with similar measure, and the variable values are normally distributed. Additionally, Pearson correlation can only capture linear relationships. Due to the fact that the assumptions of Pearson correlation are not fit with the survey data. The alternative is provided by using Spearman’s rank correlation coefficient, which is also widely used in ecological studies.

The exploration of co-occurrence networks is a useful method for determining interactions of co- occurring injuries. Network analysis has also suggested important injuries in networks. The important injuries were selected from the node features such as node degree, clustering coefficient, and betweenness. The betweenness represents the importance of the control potential that an injury exerts over the associations of other injuries in the network. The clustering coefficients are indicative of the potential spreading of the incidence of injuries through the network. As activated injury can activate other injuries, a more densely connected network facilitates injury activation \cite{Williams_2014_demonstrating}. In the network of dry season in West Java, Indonesia, BLB and SR can be the targets to be monitored because they have high betweenness, which indicated that they are more likely to present then others injuries.

It is good attempt to detect communities in a network because it can reveal information about the networks that is maybe not easy to detect by simple observation. The communities are groups of nodes that are densely connected among their node members, and slightly connected with the rest of the network.  In this study, we detected node community based on the optimization of the modularity of a sub-network, which is an approach that widely applied in many fields \cite{Liu_2014_Detecting}. Even though, the groups of injury profiles from this study are different from seasons and countries, but they are reasonable because some rice injuries present in some seasons (seasonal occurrence) such as gall midge injury \cite{Krishnaiah_2004_Rice}. The result also showed similar groups of injuries to the patterns of injuries profile from the study of \cite{Savary_2000_Characterization}. 