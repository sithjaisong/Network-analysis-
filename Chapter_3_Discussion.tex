\subsection{Discussion}

we explored the characteristics of rice injury profiles

Rice injuries were found commonly in South and South east Asia, but different levels of incidence.

%discussion on the survey data
Crop injuries that appear to be near-omnipresent are: SHR, SHB, BS, PH, RWM, LF, DH, WH, and particularly, WA and WB, while others are not: BLB, SR, LB, NB. Rice tungro disease and rat injuries injuries, which were included in the survey procedure at all sites, were not considered in the analyses because of their low prevalence and their site-specificity. 

One striking result of this study is the comparatively low prevalence of the most commonly cited and studied rice disease: leaf blast. This leads to two interpretations: and (ii) this disease heavily depends on climatic conditions to develop strong epidemics, which were not encountered in all sites or seasons covered by the analysis. The survey indeed covered environments where strong blast epidemics should be expected, but they did not occur, primarily because of the regular release and deployment of resistant varieties. Similar reasoning could be applied to several other injuries, such as BLB, NB (both with a wide range of prevalences), or even PH (which exhibits generally high prevalences).

%discussion on the spearman coefficents
Co-occurrence networks were various depending on seasons and locations (countries). The number of associated injuries (nodes) and pairwise co-occurrence (edges) presenting in the networks are different.

With regard to pest management development, understanding that the relationships of rice injuries could simplify pest management decisions. As applied here, the node degree, clustering coefficients, and betweenness described the interaction of injuries in the networks. Therefore, they may be useful in finding a good indicators for monitoring pest in rice fields. High-betweenness node High- clustering coefficient node 



Searing community structure in complex network can reveal hidden information about complex networks that is maybe not easy to detect by simple observation such as nodes, which are cluse. In this study, we detected node community based on the optimization of the modularity of a sub-network, which is a popular approach \cite{Liu_2014_Detecting}. Even though, the groups of injury profiles from this study are different from seasons and countries, but they are reasonable because some rice injuries present in some seasons (seasonal occurrence) such as gall midge injury (silver shoot) \cite{Krishnaiah_2004_Rice}. from the groups of injury profile from \cite{Savary_2000_Characterization}. But there 


Our results demonstrated the topology of the network could reflect the associations between injuries, the betweenness represents the importance of the control potential that an injury exerts over the associations of other injuries in that network. The clustering coefficients are indicative of the potential spreading of the incidence of injuries through the network. As activated injury can activate other injuries, a more densely connected network facilitates injury activation. 

Based on the network analysis, we will consider the network topology to guide us to design the pest management.

The network analysis reveal

The benefit of community detection 

Communities are groups that are densely connected among their members, and sparsely connected with the rest of the network. Community structure can reveal abundant hidden information about complex networks that is not easy to detect by simple observation. 

There are ample possibilities to develop both statistically and biological sensible indicators of co-occurrence relationships of injury profiles


Network analysis has also suggested important injuries in co-occurrence networks. The important injuries were selected from the topological features such as node degree, clustering coefficient, and betweenness. These features have been used for 

The betweenness represents the importance of the control potential that an injury exerts over the associations of other injuries in that network. The clustering coefficients are indicative of the potential spreading of the incidence of injuries through the network. As activated injury can activate other injuries, a more densely connected network facilitates injury activation. \cite{Williams_2014_demonstrating}


The BLB and SR can be the targets to be monitored because they have high betweenness. SR is also best to be the target because it also associated with the RS, which has high clustering coefficients also because RS is potentially able to be co-found with many other injuries.  BLB also show the association with the injuries, which present high clustering coefficient. Both of BLB and SR are intermediate clustering coefficient., so they are formed concurrence between groups, but less potentially to present complex association. DH and WH are less associated with other injuries, so they are difficult to employ other injuries to. 

It is the good to monitored WM and RB  and SHB also even the GLH has no betweenness, but it has high clustering coefficients, and connected with two high betweenness node. It would be also good target to monitored also because when it present, it also frequency appear together. 




