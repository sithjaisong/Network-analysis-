\section*{Introduction}
In nature, growing plants can frequently be infected at the same time by more than one species of pests and pathogens. These plant enemies caused injuries, which possibly affect yield productions when they are severe. Crop health is highlighted and implemented to manage a combination of injuries, so-called injury profiles. The co-occurrence patterns of injury are beginning to provide important insight into the injury profiles, which possibly present co-occurring or anti-co-occurring relationships between injury-injury. Uncovering these patterns might have important to implications in plant disease epidemiology and management. However, there are only a few reports of injury–injury interactions in crop systems and the mechanisms of interactions are currently unknown and this could be a difficult task since complex patterns of injury profiles are related to environmental conditions, cultural practices, and geography \citep{willocquet2004research}.

To address this issue, we use in-field surveys as a useful tool to develop ground-truth databases that allow one identify actual constraints due to pests in an agricultural productions system. These sorts of databases provide an overview of the complex relationships between the crop, its management, pest injuries, yields. Several previous studies \citep{Savary:2000char, savary2000quanti, savary2005multiple, dong2010characterization, Reddy:2011hl} involved surveys that have been used to identify relationships in an individual production situation (a set of factors including cultural practices, weather condition, socioeconomics, \textit{etc} that determine agricultural production ) and the injury profiles using nonparametric multivariate analysis such as cluster analysis, correspondence analysis, multiple correspondence analysis. Performing correspondence analysis \citep{savary1997new}, they characterized the relationships between categorized levels of variables: actual yield, production situations, and injuries profiles. For example, stem rot and sheath blight are frequently found together with high (mineral) fertilizer inputs, low pesticide use, and good water management in transplanted rice crops and overall, their results led to the conclusions that observed injuries profiles were strongly associated with production situations and the level of actual yields. 

We applied the technique from ecological study the co-occurrence analysis and network theory to reveal the patterns of co-occurrence of injury profiles. While existing statistic approach for analysis the survey data e.g. cluster analysis, correspondence analysis, multiple correspondence analysis, the methods were present in this papers has the advantages of ........

Network theory is the study of relationships between entities (`nodes') and connections between these entities (`edges'). Network theory has previously been used effectively to describe social and biological datasets, and it has been shown to be a useful tool for ???. Here, we consider pest injuries as nodes and create an edge between any two injuries if they are co-existing. We give an edge greater weight if the two injuries have strong co-occurrence at either end.  The relationships will be more complex when the number of their components increased. A way to systemically model and intuitively interpret such relationships is the depiction as a graph or network. This approach has been widely used and proven very useful in biological studies \citep{Lefebvre:2011fo}. Networks typically consist of nodes, usually representing components, while links between the nodes depict their interactions \citep{PROULX:2005hx}. A correlation network is a type of network in which two nodes are connected if their respective correlation lies above a certain threshold. The construction of this network is obtained from pairwise correlation methods \citep{Toubiana:2013cv}. By using appropriate correlation measure, correlation networks can capture biologically meaningful relationships, and discover valuable information in crop health surveys.

The aim of this study was to analyze how the structure of correlation network of pest injury co-occurrence are different over 5 locations under investigation (West Java, Indonesia, ??, India, Laguna, Philippines, Central plain, Thailand, and Makhong river delta, Vietnam). By quantifying the important aspects of the position of the specific pest injury, information based on network analysis could help to understand the formation and characteristics of co-occurrence patterns. Furthermore, key factors for could be identified with this additional information