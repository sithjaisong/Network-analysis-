\section{Using network analysis to examine co-occurrence patterns of animal pests and disease in farmers' fields in irrigated lowland rice growing areas in South and South East Asia}


\subsection{Abstract}
A farmers’ field survey is useful to identify the major pest problems, the locations and severities of the infestation, and conditions that may contribute to pest problems. Co-occurrence patterns are used in this study to explore the associations between rice injuries under farmers’ field condition. Here, I gathered the survey data from 415 farmers’ fields from 5 irrigated low land rice –growing areas (West Java, Indonesia, Tami Nadu, Indonesia, Laguna, Philippines, Suphan Buri, Thailand, and Mekong River Delta, Vietnam) in both of dry and wet season. The co-occurrence networks of rice injuries were constructed and analyzed network properties by cropping seasons and locations. The structures and topological features of networks suggested the nodes that are potentially indicators of favorable conditions that can also be favorable condition for others. Node degree, clustering coefficient, and betweenness were used to quantify the nodes. Leaffolder injury was a key indicator on dry season, whereas Stem rot, leaffolder and deadheart are good indictors in wet season. In West java, Indonesia, the good indicators are bacterial leaf blight, brown spot and stem rot in dry season, and for wet season bacterial leaf streak, whorl maggot injuries and brown spot are the good indicators. At Laguna, Philippines, whorl maggot injury often presented in dry season, and also is good indicators, whereas in wet season the good indicators are whorl maggot injury, sheath blight and leaf blast. In Suphan Buri, Thailand, Brown spot and deadheart are good indicators, for wet season the occurrence of rice injuries are various, sheath rot , brown spot red stripe, narrow brown spot, and whorl maggot injury. In area of Mekong River Delta, bacterial leaf blight, brown spot and whitehead, whereas in wet season rat injury present sheath blight rice bug and narrow brown spot.

