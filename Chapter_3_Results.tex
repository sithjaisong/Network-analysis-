
\subsection{Results}

\subsubsection{Crop health survey data}

Prior to using the crop health survey data to construct networks, some exploratory plots of the data are presented and examined. The figure shows the distribution of the data. A total of 415 farmers' fields was surveyed in Tamil Nadu, India (TMN); West Java; Indonesia (WJV), Laguna, Philippines (LAG); Suphanburi, Thailand (SPB) and Mekong river delta, Vietnam (MKD) from 2009 to 2013. The data collected pertain to rice injuries caused by animal pests, and pathogens. The protocol gives more emphasis on the nature of injuries and not on the causal organism. The levels of injuries during a cropping season were computed or summarized according to the nature of injuries and thus varied in scale. For example, foliar injuries, diseases on leaves, and insect pests were summarized as area under the disease incidence progress curve, and tiller injuries and diseases as a maximum incidence. 

The injuries caused by animal pests observed during the survey period were rat injury (RT), deadheart (DH) and whitehead (WH) caused by stem borers, whorl maggot injury (WM), leaffolder injury (LF), gall midge injury or silver shoot (GM). Rat injuries were observed at all 5 survey locations with low incidence ( less than 20\% incidence). We could observe 75\% incidence of the rat injury in MKD in dry season. They were also observed at WJV in dry season, and both season in TMN, LAG, SPB. Gall midge injuries during survey period were not observed in TMN and LAG, but it was found in SPB, MKD, but WJV at 25 \% incidence. Deadheart were observed all survey sites, and it was severe in dry season at SPB and MKD, but in WJV and LAG, it was severe in dry season. The trend of whitehead incidences observed was opposite the deadheart incidence, which whitehead incidences were more server in wet season at SPB and MKD, but less severe at WJV and LAG. Leaffolder injury was observed all survey locations. The leaffolder incidences were more severe in wet season the dry season at WJV, TMN, and LAG. As apposite to SPB and MKD, they were more severe in dry season than wet season. Whorl maggot injury was observed at all locations. Mostly, they were more severe in wet season than dry season at all surveyed locations, except LAG.

The disease we recorded were bacterial leaf blight (BLB), bacterial leaf streak (BLS), brown spot (BS), leaf blast (LB), narrow brown spot (NBS), read stripe (RS), sheath blight (SHB), sheath rot (SR), false smut (FS), stem rot (SR). Rice diseases we observed in this study were commonly found at all locations, but there were some diseases that could not find especially in TMN such as BLS, BS, NBS, DP, RS, SHR, and SR. BLB was observed all location. BLB incidence was higher in dry season than wet season in WJV and MKD except in TMN, LAG, SPB. Wet season was more favorable for BLS than dry season because the incidence was higher in wet season than dry season. As same as BLS, BS incidence was higher in wet season than dry season, and it was severe in SPB. Even though, LB is common disease in these survey locations, but there were some farmer’s fields observed LB. In WJV and MKD, the LB incidence was higher in dry than wet season season. There were many field in SPB found high level of NBS incidence. Like BLS and BS, NBS incidence was more severe in wet season than dry season. There were many fields found high RS incidences. The highest incidences of DP were found in MKD. FS were commonly found at all location. The high incidents were observed in SPB, and MKD. NB were observed at all location. They were found many observations in TMN.  SHB commonly was found all location, and high incidence was in TMN, and LAG. 

Brown planthoppers (BPH) were found all location of survey sites. During the survey period, they were found higher population in wet season than dry season in TMN and MKD, but there was some observation of BPH at other locations. White backed planthoppers (WPH) were commonly found at MKD in both of dry and wet season, and there are a few observations in WJV and LAG, but at TMN and SPB, they were not observed. Rice bug (RB) could be observed all survey location. They were highly found in dry season than wet season at WJV, LAG, and MKD. but in SPB, they were found only in wet season during the survey period.  Green leafhoppers were observed all location. They were found in LAG higher than other locations.


A global network of microbial co-occurrence and mutual exclusion within and among body site niches of the human microbiome

%Global properties of the microbiome-wide network of microbial associations are summarized in Figures 2 and 3. A dominant characteristic of the network was its habitat-specific modularity. After grouping the 18 body sites into five broad areas (oral, skin, nasal, urogenital, and gut), the large majority of edges were found clustered within body areas (98.54%), and these clusters were sparsely connected through a minority of edges (1.46%). This is confirmed by the network's high modularity coefficient of 0.28 (as defined by [32]) and Markov clustering of the network (see Methods and Figure S2). It has long been observed that sites within the human microbiome are distinct in terms of microbial composition [33], and this proved to be true of microbial interactions as well: microbial relationships within each body area's community were largely unique (Table 2). The microstructure of interaction patterns - and thus in the underlying ecology - was different for different areas, however. For example, all vaginal sites within the urogenital area were interrelated in a single homogeneous community, whereas interactions within the oral cavity suggested microbial cross-talk among three distinct habitats [34]. This can be observed quantitatively based on the proportions of microbial interactions spanning body sites within each area, e.g. 69.57% among the vaginal sites and 53.19% among the oral sites, both exceeding the microbiome-wide baseline. The skin was further unique in that the large amount (57.65%) of its associations related microbes in corresponding left and right body sites (left and right antecubital fossae and retroauricular creases), reflecting consistent maintenance of bilateral symmetry in the skin microbiome.


\subsubsection*{Co  occurrence of injury profiles based on network analysis}

Co-occurrence correlations of injuries were explored using network inference based on significant correlations using non-parametric Spearman’s rank coefficient at $P$ < 0.05. The setup of Spearman's coefficient cutoff at significant correlation coefficient at at $P$ < 0.05 could efficiency reduce the sparse correlation and highlight the significant correlation between variables. The networks were determined for the co-occurrence analysis of rice injuries based on the survey data. Ten networks constructed were based on the survey data grouping different location (west java; Indonesia (WJA), Tamil Nadu; India (TMN), Laguna; Philippines (LAG), Suphan Buri; Thailand (SPB), and Mekong Rive Delta; Vietnam (MKD)) and seasons (dry, and wet season). Nodes represent rice injuries, and edges represent Spearman's correlation coefficients at $P$-value < 0.05. 

Once a network has been constructed, analytic tools and measures can be used to quantify by determining the structural properties.

\includegraphics{figures/simgraph.png}

\begin{table}
\centering
\caption{Node topology}
\begin{tabular}{lllll}
Node & Node degree & Clustering coefficient & Betweenness   \\
A    & 2           & 0                      & 0             \\
B    & 6           & 0                      & 14            \\
C    & 6           & 0.06                   & 12            \\
D    & 4           & 0.17                   & 0             \\
E    & 4           & 0.17                   & 0             \\
F    & 2           & 0                      & 0            
\end{tabular}
\end{table}

Measures of node properties of networks seek to identify the most important nodes in a network. Different measures indices the different contexts for the word ``importance''. In this study, we consider 3 features of node properties. 

\begin{itemize}

\item \textbf{Node degree} defined as the number of connections of a node. Nodes with high degree can be implied that the node has many connections or relationships with other nodes. In co-occurrence network, the high degree node can be inferred that it was found together with occurs with connected nodes. For example, in fig(), node B, and node C are the highest-degree nodes in this network. Node B have the relationships with 3 nodes. We can consider node B as the important node of this network because If we remove this node, there will be two nodes (node A and F) would disappear from this network.

\item \textbf{Local clustering coefficient} is a measure of the degree to which nodes tend to cluster together. It is defined as how often a node forms a triangle with its direct neighbors, proportional to the number of potential triangles the relevant node can form with its direct neighbors. In the context of this network, local clustering coefficients is the the probability that two neighbors of connected with each other in the network. These measures are indicative of the complex forming of co-occurrence patterns through the network. As presented nodes can be observed other nodes, a more closely connected network facilitates nodes co-occurrence. The higher clustering coefficient injury has high possibility to form complex. Node D and E have high clustering coefficient. They formed the complex relationships, which Node D connected with Node E connected to Node C, and Node C also connected with Node D. It indicated that they have close relationships. In the context of co-occurrence network, these three nodes usually occur together.

\item \textbf{Betweenness} is measure the number of paths, which a node is present on the shortest path between all other nodes. Nodes with high betweenness have been shown that there are many pair-wise relationships across a network pass through the node, which is implied that the nodes has high possibility to occur. We determine high-betweenness injuries as``indicator''. 

\end{itemize}

Moreover, I inspected the community structure of the networks based on the optimal clustering algorithm. Such communities correspond to groups of injuries that are closely associated among themselves than other nodes in one others groups.

\subsubsection{Communities, Structures and compositions of co-occurrence network of rice pest injuries}
\paragraph{Tamil Nadu, India (TMN)}

Dry season network (Fig) was composed of 6 associated injuries (DH, NB, SR, FS, RB, and LF) and captured 7 associations. The network showed two groups of injury syndromes (the combination of injuries). The groups of the injury profiles corresponded to each community based on the optimal clustering algorithm. The first group is composed of DH, LF, RB and SR. Another group consists of FS and NB. Analysis of network properties revealed that LF and FS are high-betweenness nodes. They presented co-occurrence relationships within group 1 and group 2. SR and RB have high clustering coefficient. It indicated that these two injuries usually formed complex of co-occurrence relationships. As opposed to other injuries, NB and DH have low scores on all three centrality measure. Apparently, these injuries less possibly co-occur with other injuries (low betweenness), and do not have complex co-occurrences with other injuries (low degree and clustering coefficient).

Wet season network (Fig) was composed of 12 nodes (injury variables, DP SR, FS, BLB, NB, LF, GLH, RT, RB, DH, WM, BPH) with edges. Fig reveals three groups of injury profiles. DH, GLH, SR and BPH are in the group 1 (green).  FS, NB, DP, BLB is in group 2 (orange). RT, LF, RB and WM are in group 3 (purple). Top four injuries with high betweenness, DP, SR, LF, and BLB, are the members of each of group. They possibly are found co-occurrence within the group and inter-groups of injury profiles. Considered in each group, DH and GLH have high clustering coefficient in the group 1. NB has high clustering coefficient in the group 2. RT has highest clustering coefficient comparing other injuries in the group 3.  Group 2 has high clustering coefficients indicating that this group is closed clustered, and the injuries in this group formed co-occurrence.  DP and SR are important as the linkage to occur with other groups of injuries profiles because of high betweenness and degree. WM, BPH have low value of the 3 local properties. It indicated that BPH and WM were less possible to occur, and present co-occurrence patterns, and when they were observed, they were also not able to relate to many injuries. 

%=========================================
\paragraph{West Java, Indonesia (WJV)}
 
Dry season network (Fig.) composed of 20 nodes and 51 associations. The network reveals four groups of injury profiles. Group 1 (green color) include DH and RT. Group 2 (orange color) is NBS RB, FS, BLS, LB, DP, BLB, NB, BS. Group 3 (purple color) included GM, LF, BPH, SR, SHB, GLH, RS. Group 4 (pink color) include WM and WH.  The second and third group are formed closely clusters.  RT in group 1, BLB and BS in the group 2, LF of group 3, and WM in group 4 are high-betweenness nodes with intermediate clustering coefficient.  NBS, NB, RB, BLS, FS, DP of group 2 have high clustering coefficients. Compared to group 2, the injuries in group 3 have relatively smaller than. It indicated that the injuries in the group 2 are tightly formed complex co-occurrence.  DH, WH have low value of the three features, and located far from the center of the network. BPH and WM were less possible to occur, and present co-occurrence patterns, and when they were observed, they were also not able to relate to many injuries. connectivity.

Wet season network (Fig.) composed of 19 injuries (WM, LF, GLH, DH, BLS, WH, NBS, FS, DP, RS, LB, WPH, SR, SHR, SHB, NB, BS and BLB) with 25 associations. The network was loosely connected (low clustering coefficients). It reveals 3 connected groups and one isolated group of injury profiles. Group 1 (green) DH, WM, SHB, WH, GHL, LF, LB, and RS. Group 2 (orange) is composted of WPH, SR, GM, BLS and DP. Group 3 consisted of FS, NBS, SHB, BLB. Group 4, which is isolated, is composted of BS and NB. In group 1, WM is the injuries with high betweenness, and WH is the injury with high clustering coefficient. In group 2, BLS has high betweenness, and DP has high clustering coefficient. FS in group 3 node with high betweenness and high clustering coefficient. Group 1 appeared to form complex co-occurrence patterns because the average of clustering coefficient of injuries in this group are higher than other groups of injuries.  

%=================================
\paragraph{Central Luzon, Philippines (LAG)}

The dry season network reveal three clustered groups of injury profiles. Group 1 composted of WM, RB, NB, SHB, and FS. WM and RB in this group have high rank of betweenness. Group 2 consist of LB and DP. DP in this group connected with BPH (high betweenness) from group 3, which have GLH, BPH and BLB. Interestingly, SHB in group1, and GLH in group 3 have high clustering coefficients. NB, LB, FS, and BLB featured low in three of centrality. 

Figure revealed co-occurrence network of injury profiles in dry season at Laguna, the Philippines. Network resulted four groups of injury profiles, which there are three connected and one isolated. Group 1 (green) has  SHR, RS, RT, which is isolated. Group 2 (orange) has NB, RB, GLH, BLS, and DP. This group has more more complex combination than others because the clustering coefficients of injuries in this group are higher. Group 3 (purple) has LB, SHB, LB. This group is between group1 and group 4 (pink), which has NBS, LF, BS, and BLB. The position of SHB, LB and WM in the group with high betweenness, but they do not form the the complex combination of the injuries. They tend to link the co-occurrence of the first and the fourth group of injury profiles, which connected to NB in the first group and LF in the fourth group. So they are potentially good target to be monitored too. For example, when LF presented without the present of LB, SHB, WM, it is less likely NB would present, and other injuries in the first group would less likely to present neither. 


\paragraph{Suphan Buri, Thailand (SUP)}

Dry season network shows 8 injuries (DH, BS, SR, SHB, NBS, SHR, GLH, and WH) showing 12 significant relationships. The network revealed three closed cluster of co-occurrence patterns of injuries. Group 1 (green) is composted of NBS, DH, BS, and SHB. Group 2 (orange) is SHB and GLH, and group 3 (purple) is SR and WH. Group 1 is more complex than other two groups because three out of four nodes in the groups presenting high clustering coefficient. NBS, DH, SHB, and SR seem to be clustered together. BS present high betweenness in group 1, which is associated with the other two groups. SHB has high moderate degree and high betweenness, whereas it has low betweenness.  It can form complex association with the injuries in group 1 and group 2.  WH and GLH have low betweenness and node degree. Apparently, it is less easily found with other injuries, do not tend to complex combination (low clustering coefficient), or less possible for expressing co-occurrence through the network (low betweenness). 

Network of co-occurrence patterns of injury profiles wet season at Suphan Buri, Thailand revealed 18 associated injuries (DH, BS, SR, SHB, NBS, SHR, GLH, and WH), and 79 associations. Network analysis resulted three closely clustered groups. Group 1 (green) composted of NBS, FS, RT, SHB, LB, WM, RS and RB. Group 2 (orange) consist of GLH, BS, GM, SR, BLB, BLS, LF and DP. Group 3 consisted of SHR and NB. The group 1 are bigger and 3 were relatively high clustering coefficients than other two groups of injuries. RB, NBS, LF, and BS were highly associated in wet season. Even though they were not in the same group, but group 1 and 2 were close, which is indicated by the positions in the network. Interestingly, SHR in group 3 has high betweenness, which is in-between the association of injuries of group 1 and group 3. So SHB is more likely to present in wet season, and form complex association with other injuries because it connected to the high clustering coefficient groups.

\newpage

\paragraph{Makong Rive Delta, Vietnam (MKD)}

Co-occurrence network of injury profiles of dry season in Mekong Rive Delta, Vietnam presented 20 injuries (DH, BS, SR, SHB, NBS, SHR, GLH, and WH) with 61 associations. The network reveals the three groups of injury profiles. The group 1 (green) consisted of LB, DP, RB, DH, BS, and FS. The group 2 (orange) was composed of GLH, WPH, SR, WH, BPH, RS, WM. And the group 3 (purple) consisted of RT, NB, SHB, BLB, LF, and NBS. RB in the group 1 has high rank of betweenness and clustering coefficients. The betweenness of injuries in group 2 ranged low to intermediate, but their clustering coefficients are relatively high. This indicated that they are more likely to form complex association within group than between groups. BLB in group 3 and BS in group 1 have high rank of betweenness and they were associated. The average clustering coefficient of group 1 and 3 are smaller than the group 3. So group 1 and 3 are more likely to have chance to form association between the groups.

Co-occurrence network of injury profiles in wet season at Mekong River delta, Vietnam presented 21 injuries (DH, BS, SR, SHB, NBS, SHR, GLH, and WH), and 54 associations. From the structure of this network (Fig), it seems to have three clustered groups base on optimal clustering algorithm. The group 1 (green) is composed of NB, SR, BS, WPH, RS, WM, DH, GLH, FS, RT. The second group (orange) has BLB, BPH, LB, NBS, SHB, SHR, DP. The third group is the smallest groups, which has RB, LF and WH. The members within the first groups are relatively close following the layout, and have similar level of clustering coefficients. SHB and RB, RT, LB have high node degree and betweenness, which are inferred that they have possibility to occur in wet season because they shared many co-occurrence patterns with others. Even though, SHB have high level of betweenness and node degree, but intermediate clustering coefficient.  It connected to the high-betweenness injuries such as RB and RT, where are in different groups. SHB also acted like a ``bridge'', which link the injuries of the group 1 and group 2.

%% Please add the following required packages to your document preamble:
\begin{table}[!h]
\centering
    \begin{tabular}{llllllll}
    \hline
    \multirow{3}{*}{Production environment} & \multicolumn{6}{c}{year} & \multirow{3}{*}{Total} \\ \cline{2-7}
                         & \multicolumn{2}{c}{2013} & \multicolumn{2}{c}{2014} & \multicolumn{2}{c}{2015} &     \\ \cline{2-7}
                         & DS         & WS         & DS          & WS         & DS          & WS         &     \\
                        \hline
        Central Plain, Thailand           &20          & 20          & 14          &21          &15           &12            & 102  \\
        Odisha, India                     &15            & 12          & 15          & 16         & 15          & 15           & 88  \\
        Red River Delta, Vietnam                   &15            & 15          & 15          & 15           & 15            & 15           & 90   \\
        Tamil Nadu, India                 &15            & 15          &  15           & 14         & 15            & 15         & 89  \\
        West Java, Indonesia              & 15           & 15          &   14          & 15         & 15          &            & 74   \\
                        \hline
        Total           & 80          & 77         & 73         & 81         & 75          & 57         & 443   \\
        \hline               
    \end{tabular}
    \caption{Number of farmers' fields surveyed by production environment and year}
    \label{table:Survey_data}
\end{table}
%\begin{table}
\centering
\caption{rice production season}
\label{table:rice-production-season}
\begin{tabular}{lllll}
\hline
\multicolumn{1}{c}{\multirow{2}{*}{Country}} & \multicolumn{2}{c}{Dry}                                       & \multicolumn{2}{c}{Wet/Winter*}                            \\
\cline{2-5}
\multicolumn{1}{c}{}                         & \multicolumn{1}{c}{Planting} & \multicolumn{1}{c}{Harvesting} & \multicolumn{1}{c}{Planting} & \multicolumn{1}{c}{Harvest} \\
\hline
Indonesia                                    & April                        & October                        & November                     & March                       \\
India                                        & November                     & Mar                            & Jun                          & October                     \\
Philipppines                                 & November                     & April                          & May                          & Octber                      \\
Thailand                                     & January                      & June                           & August                       & December                    \\
Vietnam                                      & December                     & April                          & April                        & August      \\
\hline               
\end{tabular}
\end{table}

\begin{figure}[h]
\begin{tabular}{c}
\includegraphics[width = 1\textwidth]{figures/network1.pdf}\\
\includegraphics[width = 1\textwidth]{figures/nodepropCP_ds.pdf}
\end{tabular}
\caption{Co-occurrence network of rice injuries in dry season at Central plain of Thailand}
\end{figure}


%\begin{figure}[p!]
%\centerline{
%      \includegraphics[width = 1\textwidth]{figures/network1.pdf}
%      }
%  \caption{Co-occurrence network of rice injuries in dry season at Central plain of Thailand}
%  \ref{fig:network_CP_ds}
%\end{figure}
%
%\afterpage{\clearpage}
%\begin{figure}[p!]
%    \centering
%        \includegraphics[width = 1\textwidth]{figures/nodepropCP_ds.pdf}
%        \caption{Network properties of }
%\end{figure}

%=== Here is the temp files
