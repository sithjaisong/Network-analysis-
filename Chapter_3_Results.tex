\subsection{Results}

\subsubsection{Dataset}

Prior to using the crop health survey data to construct network, some exploratory plots of the data are constructed and examined. Figure show the distribution of the data. A total of 415 farmers' fields were surveyed in India, Indonesia, Philippines, Thailand and Vietnam from 2009 to 2010. Survey field were intensively cultivated, 97\% of which were irrigated and 98\% were planted to modern varieties. Data on crop health and production situation were collected at booting and ripening stages according to the procedure described in the IRRI Corp Health Characterization Portfolio\cite{Savary_2009_Survey}. 

% Please add the following required packages to your document preamble:
\begin{table}[]
\centering
\begin{tabular}{llllllll}
\hline
\multirow{3}{*}{Country} & \multicolumn{6}{c}{year} & \multirow{3}{*}{Total} \\ \cline{2-7}
                         & \multicolumn{2}{c}{2010} & \multicolumn{2}{c}{2011} & \multicolumn{2}{c}{2012} &                        \\ \cline{2-7}
                         & DS         & WS          & DS          & WS         & DS          & WS         &                        \\
\hline
India                    &            & 25          & 60          &            & 20          &            & 105                    \\
Indonesia                & 5          & 20          & 30          & 30         & 20          &            & 105                    \\
Philippines              &            & 20          & 20          &            &             &            & 40                     \\
Thailand                 &            & 20          &             & 45         &             & 40         & 105                    \\
Vietnam                  &            & 20          &             & 15         & 25          &            & 70                     \\
\hline
Total                    & 5          & 115         & 110         & 90         & 65          & 40         & 425    \\
\hline               
\end{tabular}
\caption{Number of farmers' fields surveyed by country and year}
\label{table:Survey_data}
\end{table}

\subsubsection*{Global network properties}

The network considering three measures of node centrality, centrality as used here is restricted to the idea of node centrality, while the term centralization is used to refer to particular properties of the graph structure as a whole (global).

Table 1 shows the overall statistics of centralities for the identified modules. Interestingly, modules in clusters from healthy  present lower centralization and higher density than the modules in the clusters from diseased, which may indicate that those modules could be more resilient to changes and the correlations among their members are high.

Next step was to identify hubs in each of the modules. The question of which network elements are the most important cannot be answered unambiguously. Ranking nodes (species) in the network is accomplished by measuring different centrality indices using different algorithms. We used three different algorithms. First, we used degree centrality, which indicates the number of connections to other nodes in the network and has been used in numerous situations. For example, in the case of protein interactions, proteins with high degree centrality are more likely to be essential than those with low values of degree centrality. Second, we utilized betweenness centrality, which indicates the relevance of a node as capable of holding together communicating nodes: the higher the value the higher the relevance of the node as an organizing regulatory node. The betweenness centrality of a node reflects the amount of control that this node exerts over the interactions of other nodes in the network. Third, we used a double screening scheme , which combine two algorithms  and has been shown to identify hubs that are missed by other algorithms. In general, highly dense modules with low network centralization included many species, all of them with large number of species with high degree centralization and betweenness centrality.

\begin{table} 
    \begin{tabular}{ c c c c c c c c c c c }
    \hline
         Country & Node & Link & Diameter & Connectivity & Geodesic & Density & Smallworldness & Centrality & Heterogeneity \\ 
         \hline
         India & 10 & 17 & 1.38 & 0.55 & 2.02 & 0.14 & 1.71 & 0.14 & 0.51 \\
         Indonesia & 22 & 48 & 1.40 & 0.46 & 2.25 & 0.06 & 1.24 & 0.07 & 0.64 \\
         Philippines & 21 & 33 & 1.85 & 0.28 & 2.26 & 0.07 & 0.82 & 0.19 & 0.89 \\
         Thailand & 18 & 63 & 1.11 & 0.47 & 1.81 & 0.15 & 1.07 & 0.20 & 0.70 \\
         Vietnam & 20 & 47 & 1.24 & 0.53 & 2.05 & 0.08 & 1.27 & 0.08 & 0.47 \\ 
     \hline  
    \end{tabular} 
    \caption{Network statistics for network graph each country}
\label{table:Network_stat}
\end{table}


