\subsection{Results}

\subsubsection{Dataset}

Prior to using the crop health survey data to construct the network, some exploratory plots of the data are constructed and examined. The figure shows the distribution of the data. A total of 415 farmers' fields was surveyed in Tamil Nadu, India (TMN); West Java; Indonesia (WJV), Laguna, Philippines (LAG); Suphanburi, Thailand (SPB) and Mekong river delta, Vietnam (MKD) from 2009 to 2013. Survey field was intensively cultivated, 97\% of which were irrigated and 98\% were planted to modern varieties. The data collected pertain to injuries caused by pathogens, animal pests. The protocol gives more emphasis on the nature of injuries and not on the causal organism. The levels of injuries during a cropping season were computed or summarized according to the nature of injuries and thus varied in scale. For example, foliar disease were summarized as area under the disease incidence progress curve, viral diseases as area under the progress curve of the field area affected, and tiller injuries or diseases as a maximum incidence. 

The injuries caused by animal pests observed during the survey period were rat injury, deadhead and whitehead caused by stem borers, whorl maggot injury, leaffolder injury, gall midge injury (silver shoot). Rat injuries were observed high incidence in MKD in dry season 75\% as an maximum level. They were also observed at WJV in dry season, and both season in TMN, LAG, SPB. Gall midge injuries (silver shoots) in during survey period were not observed in TMN and LAG. They were found in SPB, MKD, but WJV at maximum 25 \%. Deadheart were observed all survey sites. There are more severe in dry season at SPB and MKD, but for WJV and LAG, they are  more severe in dry season. The trend of whitehead incidence observed were opposite the deadheart incidence, which were more server in wet season at SPBm and MKD, but less severe at WJV and LAG. Leaffolder injuries were observed all survey locations. They are more severe in wet season at WJV, TMN, and LAG, but more severe in dry season at SPB, and MKD. Whorl maggot inquiries were observed at all locations. They highly occurred in LAG, and they are more severe in wet season than the wet season at all location except LAG, where their incidence scattered broadly in dry season.

Brown planthoppers were  found all location of survey sites. During the survey period, they are found higher population in wet season than dry season in TMN and MKD, but  there were some observation of BPH other locations. White backed planthoppers were commonly found at MKD in both of dry and wet season, and there are a few observations in WJV and LAG, but at TMN and SPB, they were not observed. Rice bug could be observed all survey location. They were highly found in dry season than wet season at WJV, LAG , and MKD. but in SPB , they were found only in wet season during the survey period.  Green leafhopper were observed all location. They were found in LAG higher than other location.


The disease that were recorded during the survey period were bacterial leaf blight,bacterial leaf streak, brown spot, leaf blast, narrow brown spot, read stripe, sheath blight, sheath rot, false smut, stem rot. BLB were observed all location but high in wet season in SPB.  BLB were higher in dry season than wet season in WJV and MKD, but higher in  wet season in TMN, LAG, SPB. BLS were observed  all location except TMN. they were higher in wet season than dry season. BS were observed inn all location except TMN. they were found higher in wet season than dry season. The highest incident were in Thailand. LB were observed at all location. they were high in MKD in drys season . Other location were found some fields. NBS were found all location except TMN during the survey period. there were many field in SPB found high level of NBS incident. NBS were more severe in wet season than dry season. RS were observed in all location except TMN. there were many fields found high RS incident. 
DP were found all location except TMN. The high incident were found in MKD. FS were commonly found all location. the hight incidents were observed in SPB, and MKD.NB were observed  at all location. They were found many observation in TMN. S HB commonly found all location.  high incidence in TMN, and LAG.  SHR except TMN, Many observations were higher the value in major observation. SR except TMN. There are a few observation presenting the SR. 


% Please add the following required packages to your document preamble:
\begin{table}[!h]
\centering
    \begin{tabular}{llllllll}
    \hline
    \multirow{3}{*}{Production environment} & \multicolumn{6}{c}{year} & \multirow{3}{*}{Total} \\ \cline{2-7}
                         & \multicolumn{2}{c}{2013} & \multicolumn{2}{c}{2014} & \multicolumn{2}{c}{2015} &     \\ \cline{2-7}
                         & DS         & WS         & DS          & WS         & DS          & WS         &     \\
                        \hline
        Central Plain, Thailand           &20          & 20          & 14          &21          &15           &12            & 102  \\
        Odisha, India                     &15            & 12          & 15          & 16         & 15          & 15           & 88  \\
        Red River Delta, Vietnam                   &15            & 15          & 15          & 15           & 15            & 15           & 90   \\
        Tamil Nadu, India                 &15            & 15          &  15           & 14         & 15            & 15         & 89  \\
        West Java, Indonesia              & 15           & 15          &   14          & 15         & 15          &            & 74   \\
                        \hline
        Total           & 80          & 77         & 73         & 81         & 75          & 57         & 443   \\
        \hline               
    \end{tabular}
    \caption{Number of farmers' fields surveyed by production environment and year}
    \label{table:Survey_data}
\end{table}

\begin{table}
\centering
\caption{rice production season}
\label{table:rice-production-season}
\begin{tabular}{lllll}
\hline
\multicolumn{1}{c}{\multirow{2}{*}{Country}} & \multicolumn{2}{c}{Dry}                                       & \multicolumn{2}{c}{Wet/Winter*}                            \\
\cline{2-5}
\multicolumn{1}{c}{}                         & \multicolumn{1}{c}{Planting} & \multicolumn{1}{c}{Harvesting} & \multicolumn{1}{c}{Planting} & \multicolumn{1}{c}{Harvest} \\
\hline
Indonesia                                    & April                        & October                        & November                     & March                       \\
India                                        & November                     & Mar                            & Jun                          & October                     \\
Philipppines                                 & November                     & April                          & May                          & Octber                      \\
Thailand                                     & January                      & June                           & August                       & December                    \\
Vietnam                                      & December                     & April                          & April                        & August      \\
\hline               
\end{tabular}
\end{table}

\subsubsection*{Co-occurrence of injury profiles based on network analysis}

The co-occurrence correlations of injury variables were further explored using network inference based on strong and significant correlations through using non-parametric Spearman’s rank coefficient. Moreover, network correlation structural properties could visually summarize lots of information, which might offer the potential for quick and easy comparisons among the variables. Therefore, the correlations of network were determined for the co-occurrence analysis of injury variables, based on the crop health data in the 10 groups, which were different country and season.

The co-occurrence correlation between injury variables were determined with the Spearman's coefficients at significant $P$ < 0.05, which indicated that the co-occurring variables have good association. The setup of Spearman's coefficient cutoff could efficiency reduce the sparse correlation and highlight the significant correlation between variables. The overall network wise topological properties of the network in each country and season were summarized in Table \ref{table:Network_stat}. 

\paragraph{Tamil Nadu, India}

Dry season network was composed of 19 injury variables, but captured different associations.The dotplot showed that LF node has degree of 4 since it is linked to 4 other nodes including DH, NB, WH, BPH. In the comparison, the SHB node has degree1 being linked to NB node only. Node with largest degree is LF, second are FS and RB. Closeness, we can find LF, FS, and RB with the high values. The co-occurrence present among these node. Les values of other nodes in network are SR, D, and NB. Another important property of the nodes is called ``betweenness''. Between ness is defined as the number of geodesic paths that pass through a node. It is number of ``times'' that any node needs to go through a give node to reach any other node by the shortest path. The betweenness of this network is similar to the node degree and closeness. LF, and FS are come out on top. Interestingly, the difference between of these two node and the rest of nodes is magnified. That is, LF and FS have much higher values of betweenness than do the most of the other nodes. The global properties of the network may also be of interest. For instance, the diameter of the network can be defined as the maximum geodesic over the network. Here we find that this distance is 3. Thus the maximum shortest oath between any two nodes is 5 edges. This path connects DH with LF and related through FS, and NB. Noted, the intermediate node tend to have small values of betweenness. Another global centality is the clustering coefficient. The clustering coefficient of entire network can be define as the propability that adjacent nodes of a node are connected. The clustering coefficient for the is 0.46 indicating that slightly less than half of all node that are linked to a specific node are also linked together.


Wet season network captured 25 associations among 19 injury variables. 

\paragraph{West Java, Indonesia}

\paragraph{Central Luzon, Philippines}

\paragraph{Suphan Buri, Thailand}

\paragraph{Makong Rive Delta, Vietnam}

\begin{landscape}
\begin{table}
\centering
\begin{tabular}{rlrrrrrrrrrrr}
  \hline
 & country\_season & Node & Edges & CEN\_BET & CEN\_CLO & CEN\_DEG & DENSITY & DIAM & AVG\_P & mavr\_path & TRANSITIVITY & mclus\_coef \\ 
  \hline
1 & idn\_ds &  20 & 51 & 0.21 & 0.34 & 0.21 & 0.27 & 2.42 & 2.19 & 1.92 & 0.65 & 0.26 \\ 
  2 & idn\_ws &  19 & 25 & 0.36 & 0.15 & 0.30 & 0.15 & 2.03 & 2.80 & 2.69 & 0.30 & 0.13 \\ 
  3 & ind\_ds &   6 & 7 & 0.40 & 0.55 & 0.33 & 0.47 & 1.46 & 1.67 & 1.62 & 0.46 & 0.37 \\ 
  4 & ind\_ws &  12 & 25 & 0.26 & 0.48 & 0.35 & 0.38 & 1.34 & 1.80 & 1.73 & 0.54 & 0.36 \\ 
  5 & phl\_ds &  10 & 14 & 0.27 & 0.47 & 0.36 & 0.31 & 1.81 & 1.91 & 1.96 & 0.42 & 0.28 \\ 
  6 & phl\_ws &  15 & 17 & 0.23 & 0.08 & 0.12 & 0.16 & 4.31 & 3.35 & 2.63 & 0.50 & 0.14 \\ 
  7 & tha\_ds &   8 & 12 & 0.22 & 0.43 & 0.29 & 0.43 & 1.38 & 1.71 & 1.68 & 0.39 & 0.38 \\ 
  8 & tha\_ws &  18 & 79 & 0.09 & 0.42 & 0.31 & 0.52 & 1.07 & 1.59 & 1.49 & 0.75 & 0.51 \\ 
  9 & vnm\_ds &  19 & 63 & 0.15 & 0.29 & 0.24 & 0.37 & 1.26 & 1.74 & 1.68 & 0.53 & 0.36 \\ 
  10 & vnm\_ws &  20 & 42 & 0.21 & 0.21 & 0.20 & 0.22 & 2.21 & 2.42 & 2.11 & 0.35 & 0.21 \\ 
   \hline
\end{tabular}
\caption{Summary of co-occurrence network properties}
\label{table:Network_stat}
\end{table}
\end{landscape}



%=== Here is the temp files



