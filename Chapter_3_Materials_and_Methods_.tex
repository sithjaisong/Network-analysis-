\subsection{Materials and Methods}

\subsubsection{Study sites, sampling and data collection}
We used data from surveys conducted in Tamil Nadu  and Kerala, India; West Java, Indonesia; Laguna, Philippines; Central Plain, Thailand; and Mekong river delta, Vietnam for consecutives (2009 - 2013). These are major irrigated lowland rice growing areas where where rice is intensively cropped growing rice at least two seasons per year per year. The survey procedure and data were based on a standardized protocol described  in, ``A survey portfolio to characterize yield-reducing factors in rice'' developed by \citet{Savarysurvey2009}.

Rice injuries were included the injuries caused by animal pests, pathogens, and also weeds, which are harmful to rice plants, and importantly considered to reduce yield productivity. The data collected at the field level from individual farmers. They were evaluated over time. Each surveyed field was regarded to be the representatives with a characteristic set of attributes, which were related to cropping practices, and environments. Although a very large number of pathogens, insects and weeds are harmful to rice, many are seldom considered to cause yield losses. Data were derived from time-dependent information on diseases, insects and weeds observed at four development stages of the growing crop: tillering, booting, early dough and maturity in such a way that they would best account for possible yield reductions. Data were grouped in the field assessment procedure according to their mechanisms. Injuries on leaves such as bacterial leaf blight (BLB), bacterial leaf streak (BLS), leaf blast (LB), brown spot (BS) were determined as a proportion of injured leaves. Injuries on tillers or hills such as stem rot, sheath rot, sheath blight, whitehead, deadheart were determined as a proportion of injured tillers or panicles. Weed infestation was determined as the percent of area are covered by any weed species above the rice canopy (WA) and the weed below the rice canopy (WB). Before analysis, data were compacted over time during crop growth. Two types of data were computed, depending on the nature of injury \citep{Savarysurvey2009}. One is an area under injury progress curve (AUIPC) used for injury variables, which present on the leaves. Another is the maximum level at any of the four observations used for injury variables that can be observed on tillers, panicles, and hills. The area under injury progress curve (AUIPC) \citep{Campbell_1990_Introduction} were calculated by the mid- point method using the following equation: The area under injury progress curve (AUIPC) was calculated by the mid- point method using the following equation::

\begin{equation}
AUIPC = \sum{\frac{1}{2(X_{i} + X_{i-1})(T_{i} - T_{i-1})}}
\end{equation}

\subsubsection{Co-occurrence analysis}

The first step in the analysis is identifying outlier observations using absolute hierarchical cluster analysis \citep{Zhang_2005_General}. After removing the outliers for analysis, we constructed a weighted co-occurrence network, which adjacent matrix contains selected correlation coefficients. We employed Spearman rank correlation since distribution of the injury data is unknown and most does not satisfy the normal distribution. We considered co-occurrence both of positive and negative correlations based on Spearman's rank based correlation between pairs of pest injuries within each dataset with the strength of relationship ($\rho$ from Spearman's correlation) represented by the correlation coefficient. The coefficients with $p$-values less than $p$ = 0.05 were considered. Negative correlations (indicative of anti-co-occurrence relationship) were also included in analysis. The \texttt{R} function \texttt{cor.test} with parameter method `Spearman' (package stats) was used for calculate Spearman's correlation coefficient ($\rho$), which is defined as the Pearson correlation coefficient between the ranked variables \cite{R_2015}.

These correlation relationships were generated for each pair of injury within each location replicate as long as both injury had incidence value greater than zero. We made a network of co-occurrence relationships within each locations based on the strength of the correlation ($\rho$ from the Spearman's correlation), and co-occurrence relationships were only included if they occurred across all locations. Though this method has been illustrated to produce some spurious co-occurrence relationships among data, this rank-based correlation statistic does not require any transformation of variables to fit assumptions of normality and may outperform Pearson’s correlations. To increase our level of stringency that may reduce the appearance of spurious co-occurrences within our networks, pairwise relationships had to be consistent across all datasets of a given ecosystem type, greatly reducing the number of co-occurrence pairs.


where $X_i$ is percentage (\%) of leaves, tillers or panicles injured due to rice pests (e.g., leaf blast, leaf folder), or number of insects (e.g., plant hoppers, leaf hoppers) per quadrat, or percentage (\%) of weed infestation (ground coverage) at the $i$th observation, $T_i$ is time in rice development stage units (dsu) on a 0 to 100 scale (10: seedling, 20: tillering, 30: stem elongation, 40: booting, 50: heading, 60: flowering, 70: milk, 80: dough, 90: ripening, 100: fully mature) at the $i$th observation and $n$ is total number of observations.


Crop health survey data were collected in 420 farmers' fields from 2010 to 2012 for wet and dry seasons in different production environments across South and South East Asia. The variables collected included patterns of cropping practices, crop growth measurement and crop management status assessments, measurements of levels of injuries caused by pests, and direct measurements of actual yields from crop cuts. This data can be classified into three groups: cropping practices, crop injuries, and actual yield measurements.

\subsubsection{Network analysis}

Network models were used to illustrate the co-occurrence patterns of pest injuries within same locations, where injuries represent nodes and the presence of a co-occurrence relationship based on correlation is represented by an edge using the igraph package \cite{igraph_2006}, where each network was the union of positive or negative correlation coefficients (less than −0.25 or greater than 0.25) that were consistent within each location. 

We were also interested in generating descriptive statistics about the network that may be important for understanding co-occurrence relationships. We produced network statistics that describe the position and connectedness of injuries within each co-occurrence network. Global network properties including the density, heterogeneity, centralization were computed by using \texttt{fundamentalNetworkConcepts} function from  WGCNA package \cite{Langfelder_2008} and for the basic properties such as number of nodes, edges can be computed by using functions from igraph package . Addtionally, we also calculated the small-worldness index of the network by using \texttt{smallworldness} function qgraph package. For the node-wise properties including node degree, which is the number of co-occurrence relationships that an injury is involved in a network, we used the \texttt{degree} function from igraph package. We also calculated betweenness scores for each node (injury) using the \texttt{betweenness} function from igraph package, which is defined by the number of paths through a focal microbial node. Additionally, we calculated clustering coefficients, and eigenvector using the \texttt{transitivity} function for comparison to other networks.

\subsubsection{Key factor analysis}

Identification key factors in a network is very useful and widely used in social science. The way to identify key factors is to compare relative values of centrality such as eigenvector centrality and betweenness. Its apparent that many measures of centrality are correlated \cite{Valente_2008}.  The residual of linear relationship between eigenvector centrality and betweeness and regress betweeness on eigenvector centrality can used as an indicators of key factors. A node or factors with higher levels of betweenness and lower eigenvector centrality can be inferred that is central to the functioning of the network. Nodes with lower levels of betweeness and higher eigenvector centrality may be inferred that they are key to the functioning of the network.